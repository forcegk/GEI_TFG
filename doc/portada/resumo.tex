%%%%%%%%%%%%%%%%%%%%%%%%%%%%%%%%%%%%%%%%%%%%%%%%%%%%%%%%%%%%%%%%%%%%%%%%%%%%%%%%
\begin{abstract}\thispagestyle{empty}

  En un mundo donde la inteligencia artificial y el \textit{big data} están experimentando una explosión en popularidad sin precedentes, cada vez es más importante poseer la capacidad económica y humana para poder crear y mantener una infraestructura capaz de satisfacer las necesidades computacionales que estas dos disciplinas requieren.

  En este Trabajo Fin de Grado se documenta la creación de un mini-supercomputador y una aplicación web con ejemplos interactivos y vídeos acerca de la computación de altas prestaciones (HPC o High Performance Computing), precisamente orientados a motivar e inspirar potenciales ingenieros, y así ayudarles a dar sus primeros pasos hacia este mundo tan interesante que es la supercomputación e ingeniería de computadores.

  \vspace*{25pt}
  \begin{segundoresumo}

  In a world where artificial intelligence and \textit{big data} are experiencing an unprecedented explosion in popularity, it is increasingly important to have both economic and human capabilities for being able to create and maintain an infrastructure capable of satisfying the computational requirements these disciplines demand.

  This dissertation documents the creation of a mini-supercomputer and a web application, including some interactive examples and videos on the subject of high performance computing (HPC). It is specifically geared towards motivating and inspiring potential engineers, and thus helping them take their first steps into supercomputing and computer engineering, both being disciplines of great interest.

  \end{segundoresumo}
\vspace*{25pt}
\begin{multicols}{2}
\begin{description}
\item [\palabraschaveprincipal:] \mbox{} \\[-20pt]
  \blindlist{itemize}[7] % substitúe este comando por un itemize
                         % que relacione as palabras chave
                         % que mellor identifiquen o teu TFG
                         % no idioma principal da memoria (tipicamente: galego)
\end{description}
\begin{description}
\item [\palabraschavesecundaria:] \mbox{} \\[-20pt]
  \blindlist{itemize}[7] % substitúe este comando por un itemize
                         % que relacione as palabras chave
                         % que mellor identifiquen o teu TFG
                         % no idioma secundario da memoria (tipicamente: inglés)
\end{description}
\end{multicols}

\end{abstract}
%%%%%%%%%%%%%%%%%%%%%%%%%%%%%%%%%%%%%%%%%%%%%%%%%%%%%%%%%%%%%%%%%%%%%%%%%%%%%%%%