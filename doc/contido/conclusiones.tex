\chapter{Conclusiones}
\label{chap:conclusiones}

\lettrine{E}{n} este último capítulo se hace un balance general de este \acrlong{tfg}, a partir del cual se extraen conclusiones, así como su relación con la titulación y posibles líneas de trabajo futuro, algunas de las cuales ya se han mencionado previamente a lo largo de todo el documento.

\section{Conclusiones}
El objetivo del \acrshort{tfg} es la creación de un mini-supercomputador con \acrlong{rpi}s, que sirva como base tangible para realizar explicaciones a personas con poca o ninguna formación en este sector del \acrshort{hpc}. Personalmente, y dejando siempre espacio para la discrepancia de los profesores que lo evalúen, pienso que este objetivo se ha cumplido con creces, puesto que el resultado ha sido un supercomputador en miniatura, con todas sus secciones bien diferenciadas, y extrapolables a un supercomputador real. 

Por otro lado, en el apartado software se ha puesto mucho énfasis en la simplicidad tanto de implementación como de mantenimiento, intentando realizar las configuraciones necesarias siempre con la mínima cantidad de dependencias y condicionantes posibles. Además, el dashboard, si bien no es una pieza de código particularmente compleja, tiene invertidas una gran cantidad de horas en diseño, especialmente de cara a la docencia. A primera vista éstas pueden no parecer evidentes, pero se notan cuando uno se fija más en detalle y emplea adecuadamente los recursos que el dashboard ofrece.

En cuanto a rendimiento y escalabilidad, se puede concluir con total seguridad que quizás, para \acrshort{hpc}, la plataforma de la frambuesa no es la más adecuada. Esto no debería coger a nadie por sorpresa, ya que no existen componentes redundantes en todo el clúster (ni posibilidad de añadirlos), hay una falta importante de aceleración por hardware y, en general, ninguna \acrlong{rpi} está diseñada para ser un ordenador especialmente eficaz en este sentido (véase especialmente el impacto del ancho de banda a la memoria principal en la sección \ref{sec:comparacion_resultados}). Sin embargo, esto no quita que los test de rendimiento se puedan ejecutar sobre esta plataforma y medir el impacto de los diversos componentes de la misma sobre el resultado final, que es, como mínimo, decepcionante.

\section{Relación con la titulación}

\section{Trabajo futuro}
