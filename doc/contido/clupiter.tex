\chapter{Clúpiter}
\label{chap:conceptos_basicos}

\lettrine{P}{ara} comprender el objetivo del proyecto y su propuesta de valor, primero es necesario entender las características del mismo, la justificación de las mismas, y el funcionamiento de sus componentes más primitivos.

\section{Nombre y Marca Corporativa}
Si bien este no es un elemento altamente importante en un proyecto de ingeniería como este, si que es conveniente y nunca está de más otorgarle personalidad mediante la creación de un nombre e isotipo para el mismo.

El nombre del clúster es \textbf{Clúpiter} (ClúPIter): una mezcla entre las palabras ``Clúster'' y el ``PI'' de la \acrlong{rpi}, que de forma muy conveniente recuerda en su segunda parte a Júpiter. Este es a su vez planeta del sistema solar junto con el (enano) Plutón, nombre del clúster del GAC\footnote{\url{http://pluton.dec.udc.es}}. Es cuanto menos paradógico que el cluster en miniatura sea Júpiter, y el real sea Plutón\dots

En cuanto a la marca del cluster, éste no puede quedar sin un logo, que se muestra a continuación:

TODO AQUÍ VA EL ISOLOGO %% INSERTAR ISOLOGO

\section{Precedentes}
Existen multitud de amateurs y amantes de la informática que como proyecto personal deciden montar un clúster de \acrlong{rpi}s, entre los que por ejemplo se encuentra Jeff Geerling\footnote{\url{https://www.youtube.com/user/geerlingguy}}.

\begin{figure}[h!]
  \centering
  \includegraphics[width=0.75\textwidth]{img/cluster-pi-home.png}
  \caption{Mini-cluster casero basado en \acrlong{rpi} \cite{geerling_intro_cluster}}
  \label{fig:cluster-pi-ejemplo}
\end{figure}

Además, en el mundo académico también ha habido aproximaciones muy similares al trabajo desarrollado, destacando \textit{Wee Archie}, un mini-supercomputador muy similar al de este proyecto de mayor tamaño, que permite explicaciones algo más interesantes con sus pantallas de matriz de LED.

Como se puede leer en la página web del proyecto \cite{wee_archie_webpage}, este es un supercomputador del tamaño de una maleta, que al igual que Clúpiter, sirve para explicar el funcionamiento de su hermano mayor, el supercomputador ARCHER\footnote{\url{https://www.archer.ac.uk}}.

\begin{figure}[h!]
  \centering
  \includegraphics[width=0.75\textwidth]{img/wee-girl.jpg}
  \caption{Mini-supercomputador Wee Archie}
  \label{fig:wee_archie_girl}
\end{figure}

\section{Conceptos Básicos}
\subsection{Clústers}
Los clústers, en concreto para este proyecto de alto rendimiento, son conjuntos de ordenadores diseñados para ofrecer altas prestaciones de cálculo.

Estos clústers pueden ser homogéneos o heterogéneos, siendo compuestos por computadores con la misma arquitectura y capacidades, o con arquitecturas o capacidades diferentes, respectivamente.

En el caso de Clúpiter, éste es un clúster muy homogéneo, ya que está compuesto por ocho mini-computadores exactamente iguales.

\subsection{Raspberry Pi 4 Model B}
Cada aproximadamente dos años, la Raspberry Pi Foundation saca una nueva versión de su compacto y más exitoso producto: la \acrlong{rpi}, o por sus siglas \acrshort{rpi}. Estos pequeños computadores vienen en diversos formatos de forma, pero el que ha catapultado esta plataforma al éxito ha sido el formato que denominan \textit{Standard}\footnote{85.6 mm × 56.5 mm}.

\begin{figure}[h!]
  \centering
  \includegraphics[width=0.75\textwidth]{img/rpi_parts/rpi_base.jpg}
  \caption{\acrlong{rpi} 4 Model B}
  \label{fig:rpi_base}
\end{figure}

Siendo cada versión mucho más potente que la anterior, y costando aproximadamente el mismo precio, está claro que la Raspberry Pi Foundation está haciendo un excelente trabajo aportando valor a este segmento del bajo consumo y coste.

La \acrlong{rpi} 4B (4 Model B) es la versión más nueva de formato standard, y por ello ha sido la elección para constituir la base del clúster. Las especificaciones de la misma son:

\subsubsection{CPU}
La CPU de este nuevo modelo es la \textbf{Broadcom BCM2711}, un procesador con arquitectura ARMv8-A y 4 núcleos Cortex-A72.

Estos núcleos cuentan con un \textit{pipeline} de 15 etapas, ejecución OOO (\textit{out-of-order}) y predictor de saltos, entre otros.

\begin{wrapfigure}[5]{r}{0.18\textwidth}
  \centering
  \includegraphics[width=0.18\textwidth]{img/rpi_parts/rpi_cpu.jpg}
  \label{fig:rpi_cpu}
\end{wrapfigure}
Además, cuenta con 48KB de L1I y 32KB de L1D por núcleo, así como 1MB de L2 compartido (256KB por núcleo). Esto emparejado con un único chip de 2GB de memoria LPDDR4-2400, es suficiente para un uso doméstico sencillo, pero quizás no sea la mejor disposición para el cómputo intensivo como se verá más adelante.
% TODO INSERTAR REF?????

\subsubsection{GPU/VPU}
A pesar de que el integrado que se muestra en la foto del apartado superior es un \acrshort{soc} (\textit{\acrlong{soc}}), se ha preferido tratar la CPU y la GPU/VPU por separado. Y es que los gráficos integrados de esta última \acrlong{rpi} son una importante mejora sobre la VideoCore IV que equipaban modelos anteriores.

La GPU en la \acrshort{rpi}4 es la VideoCore VI, y cuenta con multitud de soporte para multimedia y una potencia gráfica aceptable. En concreto, acelera por hardware H.265 (4Kp60 decode), H.264 (1080p60 decode, 1080p30 encode), y soporta OpenGL ES, 3.0.

\begin{wrapfigure}[7]{r}{0.32\textwidth}
  \centering
  \includegraphics[width=0.32\textwidth]{img/vulkan_logo.png}
  \label{fig:vulkan_logo}
\end{wrapfigure}
Además, y lo que considero más interesante, es que posteriormente al lanzamiento se comenzó para esta gráfica el desarrollo de un driver de Vulkan: un estándar abierto de última generación para programación de gráficos, pero que también se puede emplear para cómputo. Esto es una fantástica noticia, ya que permite acelerar ciertas cargas de trabajo aprovechando la inmensa capacidad de cómputo paralelo de estos dispositivos, siendo una de las más habituales en estos dispositivos la transformada de Fourier.

Realizar algún tipo de programa para computación \acrshort{gpgpu} (\textit{\acrlong{gpgpu}}) en Vulkan queda fuera de lo que pretende abarcar este trabajo, pero es un interesante proyecto para la realización de alguna otra iteración sobre esta plataforma.

\subsubsection{E/S}
En términos de entrada/salida la \acrshort{rpi}4B cuenta con multitud de puertos. Los que se van a usar son Gigabit Ethernet y USB 3.0, sin embargo equipa de serie otros muchos ortos puertos en el \acrshort{gpio} (\acrlong{gpio}), como son DSI (\textit{Display Serial Interface}), CSI (\textit{Camera Serial Interface}), I2C (\textit{Inter-Integrated Circuit}), UART (\textit{Universal Asynchronous Receiver-Transmitter}), SPI (\textit{Serial Peripheral Interface}), y video compuesto, así como dos salidas HDMI.

\begin{figure}[h!]
  \centering
  \includegraphics[width=0.85\textwidth]{img/rpi_parts/rpi_gpio.png}
  \caption{Pines GPIO de la \acrlong{rpi} 3, idénticos a los de la 4}
  \label{fig:rpi_gpio_pinout}
\end{figure}

\subsection{MPI}
\begin{wrapfigure}[5]{r}{0.19\textwidth}
  \centering
  \includegraphics[width=0.19\textwidth]{img/ompi_logo_2.png}
  \label{fig:ompi_logo}
\end{wrapfigure} 

\acrshort{mpi} (\acrlong{mpi})\footnote{\url{https://www.mpi-forum.org}} es un estándar de paso de mensajes diseñado para la computación en arquitecturas paralelas. Esto es, define la sintaxis y semántica de las rutinas que exponen las diferentes implementaciones.

Entre estas implementaciones hay algunas muy importantes y que son estándares de facto, como:
\begin{itemize}
  \item\textbf{OpenMPI}\footnote{\url{https://www.open-mpi.org}}: Implementación libre mantenida por un consorcio de académicos, investigadores y socios que ofrece elevada eficiencia y flexibilidad a quien la implementa. Además es la librería \acrshort{mpi} contra la que se compilan por defecto los ejecutables en Arch Linux.
  \item\textbf{MPICH}\footnote{\url{https://www.mpich.org}}: Otra implementación libre de MPI muy similar a OpenMPI en rendimiento, flexibilidad y modo de desarrollo.
  \item\textbf{Intel MPI}\footnote{\url{https://software.intel.com/content/www/us/en/develop/tools/oneapi/components/mpi-library.html}}: Implementación propietaria de \acrshort{mpi} por parte de Intel. Está especializada en productos de la propia marca y optimiza para ellos, por lo que suele arrojar un mayor rendimiento que las contrapartes libres.
\end{itemize}

\subsubsection{Envío de datos en \acrshort{mpi}}
\acrshort{mpi} funciona mediante el uso de \textbf{primitivas}, es decir, operaciones que realizan algún tipo de cálculo y/o transferencia de datos entre nodos de un mismo \textbf{comunicador}.

La base de \acrshort{mpi} son las primitivas de envío y recepción de datos, y existen dos funciones de envío y recepción \textbf{síncronas} (esto es, la llamada a la función solamente termina cuando todos los miembros involucrados la ejecución de dicha función han terminado).
\begin{itemize}
  \item \textbf{MPI\_Send}: Envía un mensaje a destino, y espera a que el receptor esté listo para recibirlo. 
  \item \textbf{MPI\_Recv}: Espera a que llegue un mensaje del origen, y cuando éste está listo para enviarlo, lo recibe.
\end{itemize}

Estas funciones tienen su contraparte para programación asíncrona, que puede arrojar un mayor rendimiento, pero también requiere un extra de complejidad y puede no ser trivial. Las funciones básicas para realizar programación asíncrona con \acrshort{mpi} son \textbf{MPI\_Isend} y \textbf{MPI\_Irecv}. A su vez son necesarias para chequear el estado de la ejecución asíncrona las funciones \textbf{MPI\_Test} (asíncrona) y \textbf{MPI\_Wait} (síncrona).

\subsubsection{Primitivas de grupo en \acrshort{mpi}}
Otra funcionalidad de \acrshort{mpi} ampliamente utilizada son las primitivas de grupo. La mayoría de ellas tienen en común que:
\begin{itemize}
  \item Son funciones llamadas por todos los procesos del comunicador.
  \item La comunicación es síncrona.
\end{itemize}

Las primitivas más sencillas y comunes de MPI, desde el punto de vista del proceso raíz, son las siguientes:
\begin{itemize}
  \item \textbf{MPI\_Bcast}: Emite (hace \textit{broadcast}) el mensaje a todos los procesos del comunicador.
  
  \begin{figure}[H]
    \vspace*{0.5cm}
    \centering
    \resizebox {0.8\textwidth} {!} {
    % Created by Eps2pgf 0.7.0 (build on 2008-08-24) on Sun Aug 22 14:08:15 CEST 2021
\begin{pgfpicture}
\pgfpathmoveto{\pgfqpoint{0cm}{0cm}}
\pgfpathlineto{\pgfqpoint{25.188cm}{0cm}}
\pgfpathlineto{\pgfqpoint{25.188cm}{10.724cm}}
\pgfpathlineto{\pgfqpoint{0cm}{10.724cm}}
\pgfpathclose
\pgfusepath{clip}
\begin{pgfscope}
\pgfpathmoveto{\pgfqpoint{0cm}{10.724cm}}
\pgfpathlineto{\pgfqpoint{0cm}{0cm}}
\pgfpathlineto{\pgfqpoint{25.188cm}{0cm}}
\pgfpathlineto{\pgfqpoint{25.188cm}{10.724cm}}
\pgfpathclose
\pgfusepath{clip}
\begin{pgfscope}
\begin{pgfscope}
\definecolor{eps2pgf_color}{rgb}{1,1,0}\pgfsetstrokecolor{eps2pgf_color}\pgfsetfillcolor{eps2pgf_color}
\pgfpathmoveto{\pgfqpoint{5.429cm}{5.92cm}}
\pgfpathlineto{\pgfqpoint{17.97cm}{5.92cm}}
\pgfpathlineto{\pgfqpoint{17.97cm}{4.173cm}}
\pgfpathlineto{\pgfqpoint{5.429cm}{4.173cm}}
\pgfpathclose
\pgfseteorule\pgfusepath{fill}\pgfsetnonzerorule
\end{pgfscope}
\begin{pgfscope}
\pgfsetdash{}{0cm}
\pgfsetlinewidth{0.317mm}
\definecolor{eps2pgf_color}{rgb}{1,1,0}\pgfsetstrokecolor{eps2pgf_color}\pgfsetfillcolor{eps2pgf_color}
\pgfpathmoveto{\pgfqpoint{5.429cm}{5.92cm}}
\pgfpathlineto{\pgfqpoint{17.97cm}{5.92cm}}
\pgfpathlineto{\pgfqpoint{17.97cm}{4.173cm}}
\pgfpathlineto{\pgfqpoint{5.429cm}{4.173cm}}
\pgfpathclose
\pgfusepath{stroke}
\end{pgfscope}
\begin{pgfscope}
\definecolor{eps2pgf_color}{rgb}{0.53,0.81,1}\pgfsetstrokecolor{eps2pgf_color}\pgfsetfillcolor{eps2pgf_color}
\pgfpathmoveto{\pgfqpoint{2.73cm}{9.095cm}}
\pgfpathlineto{\pgfqpoint{3.524cm}{9.095cm}}
\pgfpathlineto{\pgfqpoint{3.524cm}{8.301cm}}
\pgfpathlineto{\pgfqpoint{2.73cm}{8.301cm}}
\pgfpathclose
\pgfseteorule\pgfusepath{fill}\pgfsetnonzerorule
\end{pgfscope}
\begin{pgfscope}
\pgfsetdash{}{0cm}
\pgfsetlinewidth{0.317mm}
\definecolor{eps2pgf_color}{gray}{0}\pgfsetstrokecolor{eps2pgf_color}\pgfsetfillcolor{eps2pgf_color}
\pgfpathmoveto{\pgfqpoint{2.73cm}{9.095cm}}
\pgfpathlineto{\pgfqpoint{3.524cm}{9.095cm}}
\pgfpathlineto{\pgfqpoint{3.524cm}{8.301cm}}
\pgfpathlineto{\pgfqpoint{2.73cm}{8.301cm}}
\pgfpathclose
\pgfusepath{stroke}
\end{pgfscope}
\begin{pgfscope}
\definecolor{eps2pgf_color}{rgb}{0,1,0}\pgfsetstrokecolor{eps2pgf_color}\pgfsetfillcolor{eps2pgf_color}
\pgfpathmoveto{\pgfqpoint{3.524cm}{9.095cm}}
\pgfpathlineto{\pgfqpoint{4.318cm}{9.095cm}}
\pgfpathlineto{\pgfqpoint{4.318cm}{8.301cm}}
\pgfpathlineto{\pgfqpoint{3.524cm}{8.301cm}}
\pgfpathclose
\pgfseteorule\pgfusepath{fill}\pgfsetnonzerorule
\end{pgfscope}
\begin{pgfscope}
\pgfsetdash{}{0cm}
\pgfsetlinewidth{0.317mm}
\definecolor{eps2pgf_color}{gray}{0}\pgfsetstrokecolor{eps2pgf_color}\pgfsetfillcolor{eps2pgf_color}
\pgfpathmoveto{\pgfqpoint{3.524cm}{9.095cm}}
\pgfpathlineto{\pgfqpoint{4.318cm}{9.095cm}}
\pgfpathlineto{\pgfqpoint{4.318cm}{8.301cm}}
\pgfpathlineto{\pgfqpoint{3.524cm}{8.301cm}}
\pgfpathclose
\pgfusepath{stroke}
\end{pgfscope}
\begin{pgfscope}
\definecolor{eps2pgf_color}{rgb}{1,1,0}\pgfsetstrokecolor{eps2pgf_color}\pgfsetfillcolor{eps2pgf_color}
\pgfpathmoveto{\pgfqpoint{1.937cm}{9.095cm}}
\pgfpathlineto{\pgfqpoint{2.73cm}{9.095cm}}
\pgfpathlineto{\pgfqpoint{2.73cm}{8.301cm}}
\pgfpathlineto{\pgfqpoint{1.937cm}{8.301cm}}
\pgfpathclose
\pgfseteorule\pgfusepath{fill}\pgfsetnonzerorule
\end{pgfscope}
\begin{pgfscope}
\pgfsetdash{}{0cm}
\pgfsetlinewidth{0.317mm}
\definecolor{eps2pgf_color}{gray}{0}\pgfsetstrokecolor{eps2pgf_color}\pgfsetfillcolor{eps2pgf_color}
\pgfpathmoveto{\pgfqpoint{1.937cm}{9.095cm}}
\pgfpathlineto{\pgfqpoint{2.73cm}{9.095cm}}
\pgfpathlineto{\pgfqpoint{2.73cm}{8.301cm}}
\pgfpathlineto{\pgfqpoint{1.937cm}{8.301cm}}
\pgfpathclose
\pgfusepath{stroke}
\end{pgfscope}
\begin{pgfscope}
\definecolor{eps2pgf_color}{rgb}{1,0.75,0.75}\pgfsetstrokecolor{eps2pgf_color}\pgfsetfillcolor{eps2pgf_color}
\pgfpathmoveto{\pgfqpoint{4.318cm}{9.095cm}}
\pgfpathlineto{\pgfqpoint{5.112cm}{9.095cm}}
\pgfpathlineto{\pgfqpoint{5.112cm}{8.301cm}}
\pgfpathlineto{\pgfqpoint{4.318cm}{8.301cm}}
\pgfpathclose
\pgfseteorule\pgfusepath{fill}\pgfsetnonzerorule
\end{pgfscope}
\begin{pgfscope}
\pgfsetdash{}{0cm}
\pgfsetlinewidth{0.317mm}
\definecolor{eps2pgf_color}{gray}{0}\pgfsetstrokecolor{eps2pgf_color}\pgfsetfillcolor{eps2pgf_color}
\pgfpathmoveto{\pgfqpoint{4.318cm}{9.095cm}}
\pgfpathlineto{\pgfqpoint{5.112cm}{9.095cm}}
\pgfpathlineto{\pgfqpoint{5.112cm}{8.301cm}}
\pgfpathlineto{\pgfqpoint{4.318cm}{8.301cm}}
\pgfpathclose
\pgfusepath{stroke}
\end{pgfscope}
\begin{pgfscope}
\definecolor{eps2pgf_color}{rgb}{0.53,0.81,1}\pgfsetstrokecolor{eps2pgf_color}\pgfsetfillcolor{eps2pgf_color}
\pgfpathmoveto{\pgfqpoint{2.889cm}{1.157cm}}
\pgfpathlineto{\pgfqpoint{3.683cm}{1.157cm}}
\pgfpathlineto{\pgfqpoint{3.683cm}{0.363cm}}
\pgfpathlineto{\pgfqpoint{2.889cm}{0.363cm}}
\pgfpathclose
\pgfseteorule\pgfusepath{fill}\pgfsetnonzerorule
\end{pgfscope}
\begin{pgfscope}
\pgfsetdash{}{0cm}
\pgfsetlinewidth{0.317mm}
\definecolor{eps2pgf_color}{gray}{0}\pgfsetstrokecolor{eps2pgf_color}\pgfsetfillcolor{eps2pgf_color}
\pgfpathmoveto{\pgfqpoint{2.889cm}{1.157cm}}
\pgfpathlineto{\pgfqpoint{3.683cm}{1.157cm}}
\pgfpathlineto{\pgfqpoint{3.683cm}{0.363cm}}
\pgfpathlineto{\pgfqpoint{2.889cm}{0.363cm}}
\pgfpathclose
\pgfusepath{stroke}
\end{pgfscope}
\begin{pgfscope}
\definecolor{eps2pgf_color}{rgb}{0,1,0}\pgfsetstrokecolor{eps2pgf_color}\pgfsetfillcolor{eps2pgf_color}
\pgfpathmoveto{\pgfqpoint{3.683cm}{1.157cm}}
\pgfpathlineto{\pgfqpoint{4.477cm}{1.157cm}}
\pgfpathlineto{\pgfqpoint{4.477cm}{0.363cm}}
\pgfpathlineto{\pgfqpoint{3.683cm}{0.363cm}}
\pgfpathclose
\pgfseteorule\pgfusepath{fill}\pgfsetnonzerorule
\end{pgfscope}
\begin{pgfscope}
\pgfsetdash{}{0cm}
\pgfsetlinewidth{0.317mm}
\definecolor{eps2pgf_color}{gray}{0}\pgfsetstrokecolor{eps2pgf_color}\pgfsetfillcolor{eps2pgf_color}
\pgfpathmoveto{\pgfqpoint{3.683cm}{1.157cm}}
\pgfpathlineto{\pgfqpoint{4.477cm}{1.157cm}}
\pgfpathlineto{\pgfqpoint{4.477cm}{0.363cm}}
\pgfpathlineto{\pgfqpoint{3.683cm}{0.363cm}}
\pgfpathclose
\pgfusepath{stroke}
\end{pgfscope}
\begin{pgfscope}
\definecolor{eps2pgf_color}{rgb}{1,1,0}\pgfsetstrokecolor{eps2pgf_color}\pgfsetfillcolor{eps2pgf_color}
\pgfpathmoveto{\pgfqpoint{2.095cm}{1.157cm}}
\pgfpathlineto{\pgfqpoint{2.889cm}{1.157cm}}
\pgfpathlineto{\pgfqpoint{2.889cm}{0.363cm}}
\pgfpathlineto{\pgfqpoint{2.095cm}{0.363cm}}
\pgfpathclose
\pgfseteorule\pgfusepath{fill}\pgfsetnonzerorule
\end{pgfscope}
\begin{pgfscope}
\pgfsetdash{}{0cm}
\pgfsetlinewidth{0.317mm}
\definecolor{eps2pgf_color}{gray}{0}\pgfsetstrokecolor{eps2pgf_color}\pgfsetfillcolor{eps2pgf_color}
\pgfpathmoveto{\pgfqpoint{2.095cm}{1.157cm}}
\pgfpathlineto{\pgfqpoint{2.889cm}{1.157cm}}
\pgfpathlineto{\pgfqpoint{2.889cm}{0.363cm}}
\pgfpathlineto{\pgfqpoint{2.095cm}{0.363cm}}
\pgfpathclose
\pgfusepath{stroke}
\end{pgfscope}
\begin{pgfscope}
\definecolor{eps2pgf_color}{rgb}{1,0.75,0.75}\pgfsetstrokecolor{eps2pgf_color}\pgfsetfillcolor{eps2pgf_color}
\pgfpathmoveto{\pgfqpoint{4.477cm}{1.157cm}}
\pgfpathlineto{\pgfqpoint{5.27cm}{1.157cm}}
\pgfpathlineto{\pgfqpoint{5.27cm}{0.363cm}}
\pgfpathlineto{\pgfqpoint{4.477cm}{0.363cm}}
\pgfpathclose
\pgfseteorule\pgfusepath{fill}\pgfsetnonzerorule
\end{pgfscope}
\begin{pgfscope}
\pgfsetdash{}{0cm}
\pgfsetlinewidth{0.317mm}
\definecolor{eps2pgf_color}{gray}{0}\pgfsetstrokecolor{eps2pgf_color}\pgfsetfillcolor{eps2pgf_color}
\pgfpathmoveto{\pgfqpoint{4.477cm}{1.157cm}}
\pgfpathlineto{\pgfqpoint{5.27cm}{1.157cm}}
\pgfpathlineto{\pgfqpoint{5.27cm}{0.363cm}}
\pgfpathlineto{\pgfqpoint{4.477cm}{0.363cm}}
\pgfpathclose
\pgfusepath{stroke}
\end{pgfscope}
\begin{pgfscope}
\definecolor{eps2pgf_color}{rgb}{0.53,0.81,1}\pgfsetstrokecolor{eps2pgf_color}\pgfsetfillcolor{eps2pgf_color}
\pgfpathmoveto{\pgfqpoint{9.557cm}{1.316cm}}
\pgfpathlineto{\pgfqpoint{10.35cm}{1.316cm}}
\pgfpathlineto{\pgfqpoint{10.35cm}{0.522cm}}
\pgfpathlineto{\pgfqpoint{9.557cm}{0.522cm}}
\pgfpathclose
\pgfseteorule\pgfusepath{fill}\pgfsetnonzerorule
\end{pgfscope}
\begin{pgfscope}
\pgfsetdash{}{0cm}
\pgfsetlinewidth{0.317mm}
\definecolor{eps2pgf_color}{gray}{0}\pgfsetstrokecolor{eps2pgf_color}\pgfsetfillcolor{eps2pgf_color}
\pgfpathmoveto{\pgfqpoint{9.557cm}{1.316cm}}
\pgfpathlineto{\pgfqpoint{10.35cm}{1.316cm}}
\pgfpathlineto{\pgfqpoint{10.35cm}{0.522cm}}
\pgfpathlineto{\pgfqpoint{9.557cm}{0.522cm}}
\pgfpathclose
\pgfusepath{stroke}
\end{pgfscope}
\begin{pgfscope}
\definecolor{eps2pgf_color}{rgb}{0,1,0}\pgfsetstrokecolor{eps2pgf_color}\pgfsetfillcolor{eps2pgf_color}
\pgfpathmoveto{\pgfqpoint{10.35cm}{1.316cm}}
\pgfpathlineto{\pgfqpoint{11.144cm}{1.316cm}}
\pgfpathlineto{\pgfqpoint{11.144cm}{0.522cm}}
\pgfpathlineto{\pgfqpoint{10.35cm}{0.522cm}}
\pgfpathclose
\pgfseteorule\pgfusepath{fill}\pgfsetnonzerorule
\end{pgfscope}
\begin{pgfscope}
\pgfsetdash{}{0cm}
\pgfsetlinewidth{0.317mm}
\definecolor{eps2pgf_color}{gray}{0}\pgfsetstrokecolor{eps2pgf_color}\pgfsetfillcolor{eps2pgf_color}
\pgfpathmoveto{\pgfqpoint{10.35cm}{1.316cm}}
\pgfpathlineto{\pgfqpoint{11.144cm}{1.316cm}}
\pgfpathlineto{\pgfqpoint{11.144cm}{0.522cm}}
\pgfpathlineto{\pgfqpoint{10.35cm}{0.522cm}}
\pgfpathclose
\pgfusepath{stroke}
\end{pgfscope}
\begin{pgfscope}
\definecolor{eps2pgf_color}{rgb}{1,1,0}\pgfsetstrokecolor{eps2pgf_color}\pgfsetfillcolor{eps2pgf_color}
\pgfpathmoveto{\pgfqpoint{8.763cm}{1.316cm}}
\pgfpathlineto{\pgfqpoint{9.557cm}{1.316cm}}
\pgfpathlineto{\pgfqpoint{9.557cm}{0.522cm}}
\pgfpathlineto{\pgfqpoint{8.763cm}{0.522cm}}
\pgfpathclose
\pgfseteorule\pgfusepath{fill}\pgfsetnonzerorule
\end{pgfscope}
\begin{pgfscope}
\pgfsetdash{}{0cm}
\pgfsetlinewidth{0.317mm}
\definecolor{eps2pgf_color}{gray}{0}\pgfsetstrokecolor{eps2pgf_color}\pgfsetfillcolor{eps2pgf_color}
\pgfpathmoveto{\pgfqpoint{8.763cm}{1.316cm}}
\pgfpathlineto{\pgfqpoint{9.557cm}{1.316cm}}
\pgfpathlineto{\pgfqpoint{9.557cm}{0.522cm}}
\pgfpathlineto{\pgfqpoint{8.763cm}{0.522cm}}
\pgfpathclose
\pgfusepath{stroke}
\end{pgfscope}
\begin{pgfscope}
\definecolor{eps2pgf_color}{rgb}{1,0.75,0.75}\pgfsetstrokecolor{eps2pgf_color}\pgfsetfillcolor{eps2pgf_color}
\pgfpathmoveto{\pgfqpoint{11.144cm}{1.316cm}}
\pgfpathlineto{\pgfqpoint{11.938cm}{1.316cm}}
\pgfpathlineto{\pgfqpoint{11.938cm}{0.522cm}}
\pgfpathlineto{\pgfqpoint{11.144cm}{0.522cm}}
\pgfpathclose
\pgfseteorule\pgfusepath{fill}\pgfsetnonzerorule
\end{pgfscope}
\begin{pgfscope}
\pgfsetdash{}{0cm}
\pgfsetlinewidth{0.317mm}
\definecolor{eps2pgf_color}{gray}{0}\pgfsetstrokecolor{eps2pgf_color}\pgfsetfillcolor{eps2pgf_color}
\pgfpathmoveto{\pgfqpoint{11.144cm}{1.316cm}}
\pgfpathlineto{\pgfqpoint{11.938cm}{1.316cm}}
\pgfpathlineto{\pgfqpoint{11.938cm}{0.522cm}}
\pgfpathlineto{\pgfqpoint{11.144cm}{0.522cm}}
\pgfpathclose
\pgfusepath{stroke}
\end{pgfscope}
\begin{pgfscope}
\definecolor{eps2pgf_color}{rgb}{0.53,0.81,1}\pgfsetstrokecolor{eps2pgf_color}\pgfsetfillcolor{eps2pgf_color}
\pgfpathmoveto{\pgfqpoint{15.907cm}{1.316cm}}
\pgfpathlineto{\pgfqpoint{16.7cm}{1.316cm}}
\pgfpathlineto{\pgfqpoint{16.7cm}{0.522cm}}
\pgfpathlineto{\pgfqpoint{15.907cm}{0.522cm}}
\pgfpathclose
\pgfseteorule\pgfusepath{fill}\pgfsetnonzerorule
\end{pgfscope}
\begin{pgfscope}
\pgfsetdash{}{0cm}
\pgfsetlinewidth{0.317mm}
\definecolor{eps2pgf_color}{gray}{0}\pgfsetstrokecolor{eps2pgf_color}\pgfsetfillcolor{eps2pgf_color}
\pgfpathmoveto{\pgfqpoint{15.907cm}{1.316cm}}
\pgfpathlineto{\pgfqpoint{16.7cm}{1.316cm}}
\pgfpathlineto{\pgfqpoint{16.7cm}{0.522cm}}
\pgfpathlineto{\pgfqpoint{15.907cm}{0.522cm}}
\pgfpathclose
\pgfusepath{stroke}
\end{pgfscope}
\begin{pgfscope}
\definecolor{eps2pgf_color}{rgb}{0,1,0}\pgfsetstrokecolor{eps2pgf_color}\pgfsetfillcolor{eps2pgf_color}
\pgfpathmoveto{\pgfqpoint{16.7cm}{1.316cm}}
\pgfpathlineto{\pgfqpoint{17.494cm}{1.316cm}}
\pgfpathlineto{\pgfqpoint{17.494cm}{0.522cm}}
\pgfpathlineto{\pgfqpoint{16.7cm}{0.522cm}}
\pgfpathclose
\pgfseteorule\pgfusepath{fill}\pgfsetnonzerorule
\end{pgfscope}
\begin{pgfscope}
\pgfsetdash{}{0cm}
\pgfsetlinewidth{0.317mm}
\definecolor{eps2pgf_color}{gray}{0}\pgfsetstrokecolor{eps2pgf_color}\pgfsetfillcolor{eps2pgf_color}
\pgfpathmoveto{\pgfqpoint{16.7cm}{1.316cm}}
\pgfpathlineto{\pgfqpoint{17.494cm}{1.316cm}}
\pgfpathlineto{\pgfqpoint{17.494cm}{0.522cm}}
\pgfpathlineto{\pgfqpoint{16.7cm}{0.522cm}}
\pgfpathclose
\pgfusepath{stroke}
\end{pgfscope}
\begin{pgfscope}
\definecolor{eps2pgf_color}{rgb}{1,1,0}\pgfsetstrokecolor{eps2pgf_color}\pgfsetfillcolor{eps2pgf_color}
\pgfpathmoveto{\pgfqpoint{15.113cm}{1.316cm}}
\pgfpathlineto{\pgfqpoint{15.907cm}{1.316cm}}
\pgfpathlineto{\pgfqpoint{15.907cm}{0.522cm}}
\pgfpathlineto{\pgfqpoint{15.113cm}{0.522cm}}
\pgfpathclose
\pgfseteorule\pgfusepath{fill}\pgfsetnonzerorule
\end{pgfscope}
\begin{pgfscope}
\pgfsetdash{}{0cm}
\pgfsetlinewidth{0.317mm}
\definecolor{eps2pgf_color}{gray}{0}\pgfsetstrokecolor{eps2pgf_color}\pgfsetfillcolor{eps2pgf_color}
\pgfpathmoveto{\pgfqpoint{15.113cm}{1.316cm}}
\pgfpathlineto{\pgfqpoint{15.907cm}{1.316cm}}
\pgfpathlineto{\pgfqpoint{15.907cm}{0.522cm}}
\pgfpathlineto{\pgfqpoint{15.113cm}{0.522cm}}
\pgfpathclose
\pgfusepath{stroke}
\end{pgfscope}
\begin{pgfscope}
\definecolor{eps2pgf_color}{rgb}{1,0.75,0.75}\pgfsetstrokecolor{eps2pgf_color}\pgfsetfillcolor{eps2pgf_color}
\pgfpathmoveto{\pgfqpoint{17.494cm}{1.316cm}}
\pgfpathlineto{\pgfqpoint{18.288cm}{1.316cm}}
\pgfpathlineto{\pgfqpoint{18.288cm}{0.522cm}}
\pgfpathlineto{\pgfqpoint{17.494cm}{0.522cm}}
\pgfpathclose
\pgfseteorule\pgfusepath{fill}\pgfsetnonzerorule
\end{pgfscope}
\begin{pgfscope}
\pgfsetdash{}{0cm}
\pgfsetlinewidth{0.317mm}
\definecolor{eps2pgf_color}{gray}{0}\pgfsetstrokecolor{eps2pgf_color}\pgfsetfillcolor{eps2pgf_color}
\pgfpathmoveto{\pgfqpoint{17.494cm}{1.316cm}}
\pgfpathlineto{\pgfqpoint{18.288cm}{1.316cm}}
\pgfpathlineto{\pgfqpoint{18.288cm}{0.522cm}}
\pgfpathlineto{\pgfqpoint{17.494cm}{0.522cm}}
\pgfpathclose
\pgfusepath{stroke}
\end{pgfscope}
\begin{pgfscope}
\definecolor{eps2pgf_color}{rgb}{0.53,0.81,1}\pgfsetstrokecolor{eps2pgf_color}\pgfsetfillcolor{eps2pgf_color}
\pgfpathmoveto{\pgfqpoint{22.415cm}{1.475cm}}
\pgfpathlineto{\pgfqpoint{23.209cm}{1.475cm}}
\pgfpathlineto{\pgfqpoint{23.209cm}{0.681cm}}
\pgfpathlineto{\pgfqpoint{22.415cm}{0.681cm}}
\pgfpathclose
\pgfseteorule\pgfusepath{fill}\pgfsetnonzerorule
\end{pgfscope}
\begin{pgfscope}
\pgfsetdash{}{0cm}
\pgfsetlinewidth{0.317mm}
\definecolor{eps2pgf_color}{gray}{0}\pgfsetstrokecolor{eps2pgf_color}\pgfsetfillcolor{eps2pgf_color}
\pgfpathmoveto{\pgfqpoint{22.415cm}{1.475cm}}
\pgfpathlineto{\pgfqpoint{23.209cm}{1.475cm}}
\pgfpathlineto{\pgfqpoint{23.209cm}{0.681cm}}
\pgfpathlineto{\pgfqpoint{22.415cm}{0.681cm}}
\pgfpathclose
\pgfusepath{stroke}
\end{pgfscope}
\begin{pgfscope}
\definecolor{eps2pgf_color}{rgb}{0,1,0}\pgfsetstrokecolor{eps2pgf_color}\pgfsetfillcolor{eps2pgf_color}
\pgfpathmoveto{\pgfqpoint{23.209cm}{1.475cm}}
\pgfpathlineto{\pgfqpoint{24.003cm}{1.475cm}}
\pgfpathlineto{\pgfqpoint{24.003cm}{0.681cm}}
\pgfpathlineto{\pgfqpoint{23.209cm}{0.681cm}}
\pgfpathclose
\pgfseteorule\pgfusepath{fill}\pgfsetnonzerorule
\end{pgfscope}
\begin{pgfscope}
\pgfsetdash{}{0cm}
\pgfsetlinewidth{0.317mm}
\definecolor{eps2pgf_color}{gray}{0}\pgfsetstrokecolor{eps2pgf_color}\pgfsetfillcolor{eps2pgf_color}
\pgfpathmoveto{\pgfqpoint{23.209cm}{1.475cm}}
\pgfpathlineto{\pgfqpoint{24.003cm}{1.475cm}}
\pgfpathlineto{\pgfqpoint{24.003cm}{0.681cm}}
\pgfpathlineto{\pgfqpoint{23.209cm}{0.681cm}}
\pgfpathclose
\pgfusepath{stroke}
\end{pgfscope}
\begin{pgfscope}
\definecolor{eps2pgf_color}{rgb}{1,1,0}\pgfsetstrokecolor{eps2pgf_color}\pgfsetfillcolor{eps2pgf_color}
\pgfpathmoveto{\pgfqpoint{21.622cm}{1.475cm}}
\pgfpathlineto{\pgfqpoint{22.415cm}{1.475cm}}
\pgfpathlineto{\pgfqpoint{22.415cm}{0.681cm}}
\pgfpathlineto{\pgfqpoint{21.622cm}{0.681cm}}
\pgfpathclose
\pgfseteorule\pgfusepath{fill}\pgfsetnonzerorule
\end{pgfscope}
\begin{pgfscope}
\pgfsetdash{}{0cm}
\pgfsetlinewidth{0.317mm}
\definecolor{eps2pgf_color}{gray}{0}\pgfsetstrokecolor{eps2pgf_color}\pgfsetfillcolor{eps2pgf_color}
\pgfpathmoveto{\pgfqpoint{21.622cm}{1.475cm}}
\pgfpathlineto{\pgfqpoint{22.415cm}{1.475cm}}
\pgfpathlineto{\pgfqpoint{22.415cm}{0.681cm}}
\pgfpathlineto{\pgfqpoint{21.622cm}{0.681cm}}
\pgfpathclose
\pgfusepath{stroke}
\end{pgfscope}
\begin{pgfscope}
\definecolor{eps2pgf_color}{rgb}{1,0.75,0.75}\pgfsetstrokecolor{eps2pgf_color}\pgfsetfillcolor{eps2pgf_color}
\pgfpathmoveto{\pgfqpoint{24.003cm}{1.475cm}}
\pgfpathlineto{\pgfqpoint{24.797cm}{1.475cm}}
\pgfpathlineto{\pgfqpoint{24.797cm}{0.681cm}}
\pgfpathlineto{\pgfqpoint{24.003cm}{0.681cm}}
\pgfpathclose
\pgfseteorule\pgfusepath{fill}\pgfsetnonzerorule
\end{pgfscope}
\begin{pgfscope}
\pgfsetdash{}{0cm}
\pgfsetlinewidth{0.317mm}
\definecolor{eps2pgf_color}{gray}{0}\pgfsetstrokecolor{eps2pgf_color}\pgfsetfillcolor{eps2pgf_color}
\pgfpathmoveto{\pgfqpoint{24.003cm}{1.475cm}}
\pgfpathlineto{\pgfqpoint{24.797cm}{1.475cm}}
\pgfpathlineto{\pgfqpoint{24.797cm}{0.681cm}}
\pgfpathlineto{\pgfqpoint{24.003cm}{0.681cm}}
\pgfpathclose
\pgfusepath{stroke}
\end{pgfscope}
\begin{pgfscope}
\definecolor{eps2pgf_color}{gray}{0}\pgfsetstrokecolor{eps2pgf_color}\pgfsetfillcolor{eps2pgf_color}
\pgftext[x=3.183cm,y=10.369cm,rotate=0]{\fontsize{27.1}{32.52}\selectfont{\textrm{\textbf{P}}}}
\end{pgfscope}
\begin{pgfscope}
\definecolor{eps2pgf_color}{gray}{0}\pgfsetstrokecolor{eps2pgf_color}\pgfsetfillcolor{eps2pgf_color}
\pgftext[x=3.556cm,y=9.987cm,rotate=0]{\fontsize{21.68}{26.02}\selectfont{\textrm{\textbf{0}}}}
\end{pgfscope}
\begin{pgfscope}
\definecolor{eps2pgf_color}{gray}{0}\pgfsetstrokecolor{eps2pgf_color}\pgfsetfillcolor{eps2pgf_color}
\pgftext[x=9.85cm,y=10.369cm,rotate=0]{\fontsize{27.1}{32.52}\selectfont{\textrm{\textbf{P}}}}
\end{pgfscope}
\begin{pgfscope}
\definecolor{eps2pgf_color}{gray}{0}\pgfsetstrokecolor{eps2pgf_color}\pgfsetfillcolor{eps2pgf_color}
\pgftext[x=10.226cm,y=10.15cm,rotate=0]{\fontsize{21.68}{26.02}\selectfont{\textrm{\textbf{1}}}}
\end{pgfscope}
\begin{pgfscope}
\definecolor{eps2pgf_color}{gray}{0}\pgfsetstrokecolor{eps2pgf_color}\pgfsetfillcolor{eps2pgf_color}
\pgftext[x=16.2cm,y=10.369cm,rotate=0]{\fontsize{27.1}{32.52}\selectfont{\textrm{\textbf{P}}}}
\end{pgfscope}
\begin{pgfscope}
\definecolor{eps2pgf_color}{gray}{0}\pgfsetstrokecolor{eps2pgf_color}\pgfsetfillcolor{eps2pgf_color}
\pgftext[x=16.572cm,y=9.992cm,rotate=0]{\fontsize{21.68}{26.02}\selectfont{\textrm{\textbf{2}}}}
\end{pgfscope}
\begin{pgfscope}
\definecolor{eps2pgf_color}{gray}{0}\pgfsetstrokecolor{eps2pgf_color}\pgfsetfillcolor{eps2pgf_color}
\pgftext[x=22.55cm,y=10.21cm,rotate=0]{\fontsize{27.1}{32.52}\selectfont{\textrm{\textbf{P}}}}
\end{pgfscope}
\begin{pgfscope}
\definecolor{eps2pgf_color}{gray}{0}\pgfsetstrokecolor{eps2pgf_color}\pgfsetfillcolor{eps2pgf_color}
\pgftext[x=22.917cm,y=9.986cm,rotate=0]{\fontsize{21.68}{26.02}\selectfont{\textrm{\textbf{3}}}}
\end{pgfscope}
\begin{pgfscope}
\pgfsetdash{}{0cm}
\pgfsetlinewidth{0.317mm}
\definecolor{eps2pgf_color}{gray}{0}\pgfsetstrokecolor{eps2pgf_color}\pgfsetfillcolor{eps2pgf_color}
\pgfpathmoveto{\pgfqpoint{1.619cm}{9.412cm}}
\pgfpathlineto{\pgfqpoint{5.429cm}{9.412cm}}
\pgfpathlineto{\pgfqpoint{5.429cm}{7.983cm}}
\pgfpathlineto{\pgfqpoint{1.619cm}{7.983cm}}
\pgfpathclose
\pgfusepath{stroke}
\end{pgfscope}
\begin{pgfscope}
\definecolor{eps2pgf_color}{gray}{0}\pgfsetstrokecolor{eps2pgf_color}\pgfsetfillcolor{eps2pgf_color}
\pgftext[x=0.74cm,y=9.489cm,rotate=0]{\fontsize{27.1}{32.52}\selectfont{\textrm{\textbf{\textit{buf}}}}}
\end{pgfscope}
\begin{pgfscope}
\pgfsetdash{}{0cm}
\pgfsetlinewidth{0.317mm}
\definecolor{eps2pgf_color}{gray}{0}\pgfsetstrokecolor{eps2pgf_color}\pgfsetfillcolor{eps2pgf_color}
\pgfpathmoveto{\pgfqpoint{1.778cm}{1.475cm}}
\pgfpathlineto{\pgfqpoint{5.588cm}{1.475cm}}
\pgfpathlineto{\pgfqpoint{5.588cm}{0.046cm}}
\pgfpathlineto{\pgfqpoint{1.778cm}{0.046cm}}
\pgfpathclose
\pgfusepath{stroke}
\end{pgfscope}
\begin{pgfscope}
\definecolor{eps2pgf_color}{gray}{0}\pgfsetstrokecolor{eps2pgf_color}\pgfsetfillcolor{eps2pgf_color}
\pgftext[x=0.899cm,y=1.551cm,rotate=0]{\fontsize{27.1}{32.52}\selectfont{\textrm{\textbf{\textit{buf}}}}}
\end{pgfscope}
\begin{pgfscope}
\definecolor{eps2pgf_color}{gray}{0}\pgfsetstrokecolor{eps2pgf_color}\pgfsetfillcolor{eps2pgf_color}
\pgftext[x=3.5cm,y=2.749cm,rotate=0]{\fontsize{27.1}{32.52}\selectfont{\textrm{\textbf{P}}}}
\end{pgfscope}
\begin{pgfscope}
\definecolor{eps2pgf_color}{gray}{0}\pgfsetstrokecolor{eps2pgf_color}\pgfsetfillcolor{eps2pgf_color}
\pgftext[x=3.873cm,y=2.367cm,rotate=0]{\fontsize{21.68}{26.02}\selectfont{\textrm{\textbf{0}}}}
\end{pgfscope}
\begin{pgfscope}
\pgfsetdash{}{0cm}
\pgfsetlinewidth{0.317mm}
\definecolor{eps2pgf_color}{gray}{0}\pgfsetstrokecolor{eps2pgf_color}\pgfsetfillcolor{eps2pgf_color}
\pgfpathmoveto{\pgfqpoint{8.445cm}{1.633cm}}
\pgfpathlineto{\pgfqpoint{12.255cm}{1.633cm}}
\pgfpathlineto{\pgfqpoint{12.255cm}{0.205cm}}
\pgfpathlineto{\pgfqpoint{8.445cm}{0.205cm}}
\pgfpathclose
\pgfusepath{stroke}
\end{pgfscope}
\begin{pgfscope}
\definecolor{eps2pgf_color}{gray}{0}\pgfsetstrokecolor{eps2pgf_color}\pgfsetfillcolor{eps2pgf_color}
\pgftext[x=7.567cm,y=1.71cm,rotate=0]{\fontsize{27.1}{32.52}\selectfont{\textrm{\textbf{\textit{buf}}}}}
\end{pgfscope}
\begin{pgfscope}
\pgfsetdash{}{0cm}
\pgfsetlinewidth{0.317mm}
\definecolor{eps2pgf_color}{gray}{0}\pgfsetstrokecolor{eps2pgf_color}\pgfsetfillcolor{eps2pgf_color}
\pgfpathmoveto{\pgfqpoint{14.795cm}{1.633cm}}
\pgfpathlineto{\pgfqpoint{18.605cm}{1.633cm}}
\pgfpathlineto{\pgfqpoint{18.605cm}{0.205cm}}
\pgfpathlineto{\pgfqpoint{14.795cm}{0.205cm}}
\pgfpathclose
\pgfusepath{stroke}
\end{pgfscope}
\begin{pgfscope}
\definecolor{eps2pgf_color}{gray}{0}\pgfsetstrokecolor{eps2pgf_color}\pgfsetfillcolor{eps2pgf_color}
\pgftext[x=13.917cm,y=1.71cm,rotate=0]{\fontsize{27.1}{32.52}\selectfont{\textrm{\textbf{\textit{buf}}}}}
\end{pgfscope}
\begin{pgfscope}
\pgfsetdash{}{0cm}
\pgfsetlinewidth{0.317mm}
\definecolor{eps2pgf_color}{gray}{0}\pgfsetstrokecolor{eps2pgf_color}\pgfsetfillcolor{eps2pgf_color}
\pgfpathmoveto{\pgfqpoint{21.304cm}{1.792cm}}
\pgfpathlineto{\pgfqpoint{25.114cm}{1.792cm}}
\pgfpathlineto{\pgfqpoint{25.114cm}{0.363cm}}
\pgfpathlineto{\pgfqpoint{21.304cm}{0.363cm}}
\pgfpathclose
\pgfusepath{stroke}
\end{pgfscope}
\begin{pgfscope}
\definecolor{eps2pgf_color}{gray}{0}\pgfsetstrokecolor{eps2pgf_color}\pgfsetfillcolor{eps2pgf_color}
\pgftext[x=20.425cm,y=1.869cm,rotate=0]{\fontsize{27.1}{32.52}\selectfont{\textrm{\textbf{\textit{buf}}}}}
\end{pgfscope}
\begin{pgfscope}
\definecolor{eps2pgf_color}{gray}{0}\pgfsetstrokecolor{eps2pgf_color}\pgfsetfillcolor{eps2pgf_color}
\pgftext[x=10.168cm,y=2.908cm,rotate=0]{\fontsize{27.1}{32.52}\selectfont{\textrm{\textbf{P}}}}
\end{pgfscope}
\begin{pgfscope}
\definecolor{eps2pgf_color}{gray}{0}\pgfsetstrokecolor{eps2pgf_color}\pgfsetfillcolor{eps2pgf_color}
\pgftext[x=10.544cm,y=2.689cm,rotate=0]{\fontsize{21.68}{26.02}\selectfont{\textrm{\textbf{1}}}}
\end{pgfscope}
\begin{pgfscope}
\definecolor{eps2pgf_color}{gray}{0}\pgfsetstrokecolor{eps2pgf_color}\pgfsetfillcolor{eps2pgf_color}
\pgftext[x=16.518cm,y=2.908cm,rotate=0]{\fontsize{27.1}{32.52}\selectfont{\textrm{\textbf{P}}}}
\end{pgfscope}
\begin{pgfscope}
\definecolor{eps2pgf_color}{gray}{0}\pgfsetstrokecolor{eps2pgf_color}\pgfsetfillcolor{eps2pgf_color}
\pgftext[x=16.889cm,y=2.53cm,rotate=0]{\fontsize{21.68}{26.02}\selectfont{\textrm{\textbf{2}}}}
\end{pgfscope}
\begin{pgfscope}
\definecolor{eps2pgf_color}{gray}{0}\pgfsetstrokecolor{eps2pgf_color}\pgfsetfillcolor{eps2pgf_color}
\pgftext[x=22.868cm,y=2.749cm,rotate=0]{\fontsize{27.1}{32.52}\selectfont{\textrm{\textbf{P}}}}
\end{pgfscope}
\begin{pgfscope}
\definecolor{eps2pgf_color}{gray}{0}\pgfsetstrokecolor{eps2pgf_color}\pgfsetfillcolor{eps2pgf_color}
\pgftext[x=23.235cm,y=2.525cm,rotate=0]{\fontsize{21.68}{26.02}\selectfont{\textrm{\textbf{3}}}}
\end{pgfscope}
\begin{pgfscope}
\pgfsetdash{}{0cm}
\pgfsetlinewidth{0.317mm}
\definecolor{eps2pgf_color}{gray}{0}\pgfsetstrokecolor{eps2pgf_color}\pgfsetfillcolor{eps2pgf_color}
\pgfpathmoveto{\pgfqpoint{8.287cm}{9.571cm}}
\pgfpathlineto{\pgfqpoint{12.097cm}{9.571cm}}
\pgfpathlineto{\pgfqpoint{12.097cm}{8.142cm}}
\pgfpathlineto{\pgfqpoint{8.287cm}{8.142cm}}
\pgfpathclose
\pgfusepath{stroke}
\end{pgfscope}
\begin{pgfscope}
\pgfsetdash{}{0cm}
\pgfsetlinewidth{0.317mm}
\definecolor{eps2pgf_color}{gray}{0}\pgfsetstrokecolor{eps2pgf_color}\pgfsetfillcolor{eps2pgf_color}
\pgfpathmoveto{\pgfqpoint{14.637cm}{9.571cm}}
\pgfpathlineto{\pgfqpoint{18.447cm}{9.571cm}}
\pgfpathlineto{\pgfqpoint{18.447cm}{8.142cm}}
\pgfpathlineto{\pgfqpoint{14.637cm}{8.142cm}}
\pgfpathclose
\pgfusepath{stroke}
\end{pgfscope}
\begin{pgfscope}
\pgfsetdash{}{0cm}
\pgfsetlinewidth{0.317mm}
\definecolor{eps2pgf_color}{gray}{0}\pgfsetstrokecolor{eps2pgf_color}\pgfsetfillcolor{eps2pgf_color}
\pgfpathmoveto{\pgfqpoint{20.987cm}{9.571cm}}
\pgfpathlineto{\pgfqpoint{24.797cm}{9.571cm}}
\pgfpathlineto{\pgfqpoint{24.797cm}{8.142cm}}
\pgfpathlineto{\pgfqpoint{20.987cm}{8.142cm}}
\pgfpathclose
\pgfusepath{stroke}
\end{pgfscope}
\begin{pgfscope}
\pgfsetdash{}{0cm}
\pgfsetlinewidth{0.317mm}
\definecolor{eps2pgf_color}{gray}{0}\pgfsetstrokecolor{eps2pgf_color}\pgfsetfillcolor{eps2pgf_color}
\pgfpathmoveto{\pgfqpoint{8.604cm}{9.253cm}}
\pgfpathlineto{\pgfqpoint{9.398cm}{9.253cm}}
\pgfpathlineto{\pgfqpoint{9.398cm}{8.46cm}}
\pgfpathlineto{\pgfqpoint{8.604cm}{8.46cm}}
\pgfpathclose
\pgfusepath{stroke}
\end{pgfscope}
\begin{pgfscope}
\pgfsetdash{}{0cm}
\pgfsetlinewidth{0.317mm}
\definecolor{eps2pgf_color}{gray}{0}\pgfsetstrokecolor{eps2pgf_color}\pgfsetfillcolor{eps2pgf_color}
\pgfpathmoveto{\pgfqpoint{9.398cm}{9.253cm}}
\pgfpathlineto{\pgfqpoint{10.192cm}{9.253cm}}
\pgfpathlineto{\pgfqpoint{10.192cm}{8.46cm}}
\pgfpathlineto{\pgfqpoint{9.398cm}{8.46cm}}
\pgfpathclose
\pgfusepath{stroke}
\end{pgfscope}
\begin{pgfscope}
\pgfsetdash{}{0cm}
\pgfsetlinewidth{0.317mm}
\definecolor{eps2pgf_color}{gray}{0}\pgfsetstrokecolor{eps2pgf_color}\pgfsetfillcolor{eps2pgf_color}
\pgfpathmoveto{\pgfqpoint{10.192cm}{9.253cm}}
\pgfpathlineto{\pgfqpoint{10.985cm}{9.253cm}}
\pgfpathlineto{\pgfqpoint{10.985cm}{8.46cm}}
\pgfpathlineto{\pgfqpoint{10.192cm}{8.46cm}}
\pgfpathclose
\pgfusepath{stroke}
\end{pgfscope}
\begin{pgfscope}
\pgfsetdash{}{0cm}
\pgfsetlinewidth{0.317mm}
\definecolor{eps2pgf_color}{gray}{0}\pgfsetstrokecolor{eps2pgf_color}\pgfsetfillcolor{eps2pgf_color}
\pgfpathmoveto{\pgfqpoint{10.985cm}{9.253cm}}
\pgfpathlineto{\pgfqpoint{11.779cm}{9.253cm}}
\pgfpathlineto{\pgfqpoint{11.779cm}{8.46cm}}
\pgfpathlineto{\pgfqpoint{10.985cm}{8.46cm}}
\pgfpathclose
\pgfusepath{stroke}
\end{pgfscope}
\begin{pgfscope}
\pgfsetdash{}{0cm}
\pgfsetlinewidth{0.317mm}
\definecolor{eps2pgf_color}{gray}{0}\pgfsetstrokecolor{eps2pgf_color}\pgfsetfillcolor{eps2pgf_color}
\pgfpathmoveto{\pgfqpoint{14.954cm}{9.253cm}}
\pgfpathlineto{\pgfqpoint{15.748cm}{9.253cm}}
\pgfpathlineto{\pgfqpoint{15.748cm}{8.46cm}}
\pgfpathlineto{\pgfqpoint{14.954cm}{8.46cm}}
\pgfpathclose
\pgfusepath{stroke}
\end{pgfscope}
\begin{pgfscope}
\pgfsetdash{}{0cm}
\pgfsetlinewidth{0.317mm}
\definecolor{eps2pgf_color}{gray}{0}\pgfsetstrokecolor{eps2pgf_color}\pgfsetfillcolor{eps2pgf_color}
\pgfpathmoveto{\pgfqpoint{15.748cm}{9.253cm}}
\pgfpathlineto{\pgfqpoint{16.542cm}{9.253cm}}
\pgfpathlineto{\pgfqpoint{16.542cm}{8.46cm}}
\pgfpathlineto{\pgfqpoint{15.748cm}{8.46cm}}
\pgfpathclose
\pgfusepath{stroke}
\end{pgfscope}
\begin{pgfscope}
\pgfsetdash{}{0cm}
\pgfsetlinewidth{0.317mm}
\definecolor{eps2pgf_color}{gray}{0}\pgfsetstrokecolor{eps2pgf_color}\pgfsetfillcolor{eps2pgf_color}
\pgfpathmoveto{\pgfqpoint{16.542cm}{9.253cm}}
\pgfpathlineto{\pgfqpoint{17.335cm}{9.253cm}}
\pgfpathlineto{\pgfqpoint{17.335cm}{8.46cm}}
\pgfpathlineto{\pgfqpoint{16.542cm}{8.46cm}}
\pgfpathclose
\pgfusepath{stroke}
\end{pgfscope}
\begin{pgfscope}
\pgfsetdash{}{0cm}
\pgfsetlinewidth{0.317mm}
\definecolor{eps2pgf_color}{gray}{0}\pgfsetstrokecolor{eps2pgf_color}\pgfsetfillcolor{eps2pgf_color}
\pgfpathmoveto{\pgfqpoint{17.335cm}{9.253cm}}
\pgfpathlineto{\pgfqpoint{18.129cm}{9.253cm}}
\pgfpathlineto{\pgfqpoint{18.129cm}{8.46cm}}
\pgfpathlineto{\pgfqpoint{17.335cm}{8.46cm}}
\pgfpathclose
\pgfusepath{stroke}
\end{pgfscope}
\begin{pgfscope}
\pgfsetdash{}{0cm}
\pgfsetlinewidth{0.317mm}
\definecolor{eps2pgf_color}{gray}{0}\pgfsetstrokecolor{eps2pgf_color}\pgfsetfillcolor{eps2pgf_color}
\pgfpathmoveto{\pgfqpoint{21.304cm}{9.253cm}}
\pgfpathlineto{\pgfqpoint{22.098cm}{9.253cm}}
\pgfpathlineto{\pgfqpoint{22.098cm}{8.46cm}}
\pgfpathlineto{\pgfqpoint{21.304cm}{8.46cm}}
\pgfpathclose
\pgfusepath{stroke}
\end{pgfscope}
\begin{pgfscope}
\pgfsetdash{}{0cm}
\pgfsetlinewidth{0.317mm}
\definecolor{eps2pgf_color}{gray}{0}\pgfsetstrokecolor{eps2pgf_color}\pgfsetfillcolor{eps2pgf_color}
\pgfpathmoveto{\pgfqpoint{22.098cm}{9.253cm}}
\pgfpathlineto{\pgfqpoint{22.892cm}{9.253cm}}
\pgfpathlineto{\pgfqpoint{22.892cm}{8.46cm}}
\pgfpathlineto{\pgfqpoint{22.098cm}{8.46cm}}
\pgfpathclose
\pgfusepath{stroke}
\end{pgfscope}
\begin{pgfscope}
\pgfsetdash{}{0cm}
\pgfsetlinewidth{0.317mm}
\definecolor{eps2pgf_color}{gray}{0}\pgfsetstrokecolor{eps2pgf_color}\pgfsetfillcolor{eps2pgf_color}
\pgfpathmoveto{\pgfqpoint{22.892cm}{9.253cm}}
\pgfpathlineto{\pgfqpoint{23.685cm}{9.253cm}}
\pgfpathlineto{\pgfqpoint{23.685cm}{8.46cm}}
\pgfpathlineto{\pgfqpoint{22.892cm}{8.46cm}}
\pgfpathclose
\pgfusepath{stroke}
\end{pgfscope}
\begin{pgfscope}
\pgfsetdash{}{0cm}
\pgfsetlinewidth{0.317mm}
\definecolor{eps2pgf_color}{gray}{0}\pgfsetstrokecolor{eps2pgf_color}\pgfsetfillcolor{eps2pgf_color}
\pgfpathmoveto{\pgfqpoint{23.685cm}{9.253cm}}
\pgfpathlineto{\pgfqpoint{24.479cm}{9.253cm}}
\pgfpathlineto{\pgfqpoint{24.479cm}{8.46cm}}
\pgfpathlineto{\pgfqpoint{23.685cm}{8.46cm}}
\pgfpathclose
\pgfusepath{stroke}
\end{pgfscope}
\begin{pgfscope}
\pgfpathmoveto{\pgfqpoint{0cm}{10.724cm}}
\pgfpathlineto{\pgfqpoint{0cm}{0cm}}
\pgfpathlineto{\pgfqpoint{25.188cm}{0cm}}
\pgfpathlineto{\pgfqpoint{25.188cm}{10.724cm}}
\pgfpathclose
\pgfpathmoveto{\pgfqpoint{11.676cm}{3.166cm}}
\pgfpathlineto{\pgfqpoint{11.568cm}{3.166cm}}
\pgfpathlineto{\pgfqpoint{11.366cm}{4.442cm}}
\pgfpathlineto{\pgfqpoint{11.874cm}{4.442cm}}
\pgfpathclose
\pgfseteorule\pgfusepath{clip}\pgfsetnonzerorule
\begin{pgfscope}
\pgfsetdash{}{0cm}
\pgfsetlinewidth{0.952mm}
\definecolor{eps2pgf_color}{gray}{0}\pgfsetstrokecolor{eps2pgf_color}\pgfsetfillcolor{eps2pgf_color}
\pgfpathmoveto{\pgfqpoint{11.62cm}{7.031cm}}
\pgfpathlineto{\pgfqpoint{11.62cm}{3.221cm}}
\pgfusepath{stroke}
\end{pgfscope}
\end{pgfscope}
\begin{pgfscope}
\definecolor{eps2pgf_color}{gray}{0}\pgfsetstrokecolor{eps2pgf_color}\pgfsetfillcolor{eps2pgf_color}
\pgfpathmoveto{\pgfqpoint{11.366cm}{4.442cm}}
\pgfpathlineto{\pgfqpoint{11.62cm}{3.426cm}}
\pgfpathlineto{\pgfqpoint{11.874cm}{4.442cm}}
\pgfpathlineto{\pgfqpoint{11.366cm}{4.442cm}}
\pgfpathclose
\pgfseteorule\pgfusepath{fill}\pgfsetnonzerorule
\end{pgfscope}
\pgfsetdash{}{0cm}
\pgfsetlinewidth{0.952mm}
\definecolor{eps2pgf_color}{gray}{0}\pgfsetstrokecolor{eps2pgf_color}\pgfsetfillcolor{eps2pgf_color}
\pgfpathmoveto{\pgfqpoint{11.366cm}{4.442cm}}
\pgfpathlineto{\pgfqpoint{11.62cm}{3.426cm}}
\pgfpathlineto{\pgfqpoint{11.874cm}{4.442cm}}
\pgfpathlineto{\pgfqpoint{11.366cm}{4.442cm}}
\pgfpathclose
\pgfusepath{stroke}
\begin{pgfscope}
\pgftext[x=7.408cm,y=9.647cm,rotate=0]{\fontsize{27.1}{32.52}\selectfont{\textrm{\textbf{\textit{buf}}}}}
\end{pgfscope}
\begin{pgfscope}
\pgftext[x=13.758cm,y=9.647cm,rotate=0]{\fontsize{27.1}{32.52}\selectfont{\textrm{\textbf{\textit{buf}}}}}
\end{pgfscope}
\begin{pgfscope}
\pgftext[x=20.108cm,y=9.647cm,rotate=0]{\fontsize{27.1}{32.52}\selectfont{\textrm{\textbf{\textit{buf}}}}}
\end{pgfscope}
\begin{pgfscope}
\definecolor{eps2pgf_color}{rgb}{1,0,0}\pgfsetstrokecolor{eps2pgf_color}\pgfsetfillcolor{eps2pgf_color}
\pgftext[x=11.264cm,y=5.053cm,rotate=0]{\fontsize{27.1}{32.52}\selectfont{\textrm{\textbf{MPI{\_}Bcast(buf, ..., 0, ... );}}}}
\end{pgfscope}
\end{pgfscope}
\end{pgfscope}
\end{pgfpicture}

    }
    \caption{Visualización de MPI\_Bcast \cite{cheung_mpi}}
    \label{fig:mpi_bcast}
  \end{figure}

  \item \textbf{MPI\_Scatter}: Distribuye equitativamente el mensaje entre todos los procesos del comunicador.

  \begin{figure}[H]
    \vspace*{0.5cm}
    \centering
    \resizebox {0.8\textwidth} {!} {
    % Created by Eps2pgf 0.7.0 (build on 2008-08-24) on Sun Aug 22 14:18:39 CEST 2021
\begin{pgfpicture}
\pgfpathmoveto{\pgfqpoint{0cm}{0cm}}
\pgfpathlineto{\pgfqpoint{25.329cm}{0cm}}
\pgfpathlineto{\pgfqpoint{25.329cm}{10.76cm}}
\pgfpathlineto{\pgfqpoint{0cm}{10.76cm}}
\pgfpathclose
\pgfusepath{clip}
\begin{pgfscope}
\pgfpathmoveto{\pgfqpoint{0cm}{10.76cm}}
\pgfpathlineto{\pgfqpoint{0cm}{0cm}}
\pgfpathlineto{\pgfqpoint{25.329cm}{0cm}}
\pgfpathlineto{\pgfqpoint{25.329cm}{10.76cm}}
\pgfpathclose
\pgfusepath{clip}
\begin{pgfscope}
\begin{pgfscope}
\definecolor{eps2pgf_color}{rgb}{1,1,0}\pgfsetstrokecolor{eps2pgf_color}\pgfsetfillcolor{eps2pgf_color}
\pgfpathmoveto{\pgfqpoint{1.778cm}{5.92cm}}
\pgfpathlineto{\pgfqpoint{20.669cm}{5.92cm}}
\pgfpathlineto{\pgfqpoint{20.669cm}{4.173cm}}
\pgfpathlineto{\pgfqpoint{1.778cm}{4.173cm}}
\pgfpathclose
\pgfseteorule\pgfusepath{fill}\pgfsetnonzerorule
\end{pgfscope}
\begin{pgfscope}
\pgfsetdash{}{0cm}
\pgfsetlinewidth{0.317mm}
\definecolor{eps2pgf_color}{rgb}{1,1,0}\pgfsetstrokecolor{eps2pgf_color}\pgfsetfillcolor{eps2pgf_color}
\pgfpathmoveto{\pgfqpoint{1.778cm}{5.92cm}}
\pgfpathlineto{\pgfqpoint{20.669cm}{5.92cm}}
\pgfpathlineto{\pgfqpoint{20.669cm}{4.173cm}}
\pgfpathlineto{\pgfqpoint{1.778cm}{4.173cm}}
\pgfpathclose
\pgfusepath{stroke}
\end{pgfscope}
\begin{pgfscope}
\definecolor{eps2pgf_color}{rgb}{1,1,0}\pgfsetstrokecolor{eps2pgf_color}\pgfsetfillcolor{eps2pgf_color}
\pgfpathmoveto{\pgfqpoint{2.254cm}{1.157cm}}
\pgfpathlineto{\pgfqpoint{3.048cm}{1.157cm}}
\pgfpathlineto{\pgfqpoint{3.048cm}{0.363cm}}
\pgfpathlineto{\pgfqpoint{2.254cm}{0.363cm}}
\pgfpathclose
\pgfseteorule\pgfusepath{fill}\pgfsetnonzerorule
\end{pgfscope}
\begin{pgfscope}
\pgfsetdash{}{0cm}
\pgfsetlinewidth{0.317mm}
\definecolor{eps2pgf_color}{gray}{0}\pgfsetstrokecolor{eps2pgf_color}\pgfsetfillcolor{eps2pgf_color}
\pgfpathmoveto{\pgfqpoint{2.254cm}{1.157cm}}
\pgfpathlineto{\pgfqpoint{3.048cm}{1.157cm}}
\pgfpathlineto{\pgfqpoint{3.048cm}{0.363cm}}
\pgfpathlineto{\pgfqpoint{2.254cm}{0.363cm}}
\pgfpathclose
\pgfusepath{stroke}
\end{pgfscope}
\begin{pgfscope}
\definecolor{eps2pgf_color}{rgb}{0.53,0.81,1}\pgfsetstrokecolor{eps2pgf_color}\pgfsetfillcolor{eps2pgf_color}
\pgfpathmoveto{\pgfqpoint{8.922cm}{1.316cm}}
\pgfpathlineto{\pgfqpoint{9.715cm}{1.316cm}}
\pgfpathlineto{\pgfqpoint{9.715cm}{0.522cm}}
\pgfpathlineto{\pgfqpoint{8.922cm}{0.522cm}}
\pgfpathclose
\pgfseteorule\pgfusepath{fill}\pgfsetnonzerorule
\end{pgfscope}
\begin{pgfscope}
\pgfsetdash{}{0cm}
\pgfsetlinewidth{0.317mm}
\definecolor{eps2pgf_color}{gray}{0}\pgfsetstrokecolor{eps2pgf_color}\pgfsetfillcolor{eps2pgf_color}
\pgfpathmoveto{\pgfqpoint{8.922cm}{1.316cm}}
\pgfpathlineto{\pgfqpoint{9.715cm}{1.316cm}}
\pgfpathlineto{\pgfqpoint{9.715cm}{0.522cm}}
\pgfpathlineto{\pgfqpoint{8.922cm}{0.522cm}}
\pgfpathclose
\pgfusepath{stroke}
\end{pgfscope}
\begin{pgfscope}
\definecolor{eps2pgf_color}{rgb}{0,1,0}\pgfsetstrokecolor{eps2pgf_color}\pgfsetfillcolor{eps2pgf_color}
\pgfpathmoveto{\pgfqpoint{15.272cm}{1.316cm}}
\pgfpathlineto{\pgfqpoint{16.065cm}{1.316cm}}
\pgfpathlineto{\pgfqpoint{16.065cm}{0.522cm}}
\pgfpathlineto{\pgfqpoint{15.272cm}{0.522cm}}
\pgfpathclose
\pgfseteorule\pgfusepath{fill}\pgfsetnonzerorule
\end{pgfscope}
\begin{pgfscope}
\pgfsetdash{}{0cm}
\pgfsetlinewidth{0.317mm}
\definecolor{eps2pgf_color}{gray}{0}\pgfsetstrokecolor{eps2pgf_color}\pgfsetfillcolor{eps2pgf_color}
\pgfpathmoveto{\pgfqpoint{15.272cm}{1.316cm}}
\pgfpathlineto{\pgfqpoint{16.065cm}{1.316cm}}
\pgfpathlineto{\pgfqpoint{16.065cm}{0.522cm}}
\pgfpathlineto{\pgfqpoint{15.272cm}{0.522cm}}
\pgfpathclose
\pgfusepath{stroke}
\end{pgfscope}
\begin{pgfscope}
\definecolor{eps2pgf_color}{rgb}{1,0.75,0.75}\pgfsetstrokecolor{eps2pgf_color}\pgfsetfillcolor{eps2pgf_color}
\pgfpathmoveto{\pgfqpoint{21.78cm}{1.475cm}}
\pgfpathlineto{\pgfqpoint{22.574cm}{1.475cm}}
\pgfpathlineto{\pgfqpoint{22.574cm}{0.681cm}}
\pgfpathlineto{\pgfqpoint{21.78cm}{0.681cm}}
\pgfpathclose
\pgfseteorule\pgfusepath{fill}\pgfsetnonzerorule
\end{pgfscope}
\begin{pgfscope}
\pgfsetdash{}{0cm}
\pgfsetlinewidth{0.317mm}
\definecolor{eps2pgf_color}{gray}{0}\pgfsetstrokecolor{eps2pgf_color}\pgfsetfillcolor{eps2pgf_color}
\pgfpathmoveto{\pgfqpoint{21.78cm}{1.475cm}}
\pgfpathlineto{\pgfqpoint{22.574cm}{1.475cm}}
\pgfpathlineto{\pgfqpoint{22.574cm}{0.681cm}}
\pgfpathlineto{\pgfqpoint{21.78cm}{0.681cm}}
\pgfpathclose
\pgfusepath{stroke}
\end{pgfscope}
\begin{pgfscope}
\definecolor{eps2pgf_color}{rgb}{0.53,0.81,1}\pgfsetstrokecolor{eps2pgf_color}\pgfsetfillcolor{eps2pgf_color}
\pgfpathmoveto{\pgfqpoint{2.889cm}{9.095cm}}
\pgfpathlineto{\pgfqpoint{3.683cm}{9.095cm}}
\pgfpathlineto{\pgfqpoint{3.683cm}{8.301cm}}
\pgfpathlineto{\pgfqpoint{2.889cm}{8.301cm}}
\pgfpathclose
\pgfseteorule\pgfusepath{fill}\pgfsetnonzerorule
\end{pgfscope}
\begin{pgfscope}
\pgfsetdash{}{0cm}
\pgfsetlinewidth{0.317mm}
\definecolor{eps2pgf_color}{gray}{0}\pgfsetstrokecolor{eps2pgf_color}\pgfsetfillcolor{eps2pgf_color}
\pgfpathmoveto{\pgfqpoint{2.889cm}{9.095cm}}
\pgfpathlineto{\pgfqpoint{3.683cm}{9.095cm}}
\pgfpathlineto{\pgfqpoint{3.683cm}{8.301cm}}
\pgfpathlineto{\pgfqpoint{2.889cm}{8.301cm}}
\pgfpathclose
\pgfusepath{stroke}
\end{pgfscope}
\begin{pgfscope}
\definecolor{eps2pgf_color}{rgb}{0,1,0}\pgfsetstrokecolor{eps2pgf_color}\pgfsetfillcolor{eps2pgf_color}
\pgfpathmoveto{\pgfqpoint{3.683cm}{9.095cm}}
\pgfpathlineto{\pgfqpoint{4.477cm}{9.095cm}}
\pgfpathlineto{\pgfqpoint{4.477cm}{8.301cm}}
\pgfpathlineto{\pgfqpoint{3.683cm}{8.301cm}}
\pgfpathclose
\pgfseteorule\pgfusepath{fill}\pgfsetnonzerorule
\end{pgfscope}
\begin{pgfscope}
\pgfsetdash{}{0cm}
\pgfsetlinewidth{0.317mm}
\definecolor{eps2pgf_color}{gray}{0}\pgfsetstrokecolor{eps2pgf_color}\pgfsetfillcolor{eps2pgf_color}
\pgfpathmoveto{\pgfqpoint{3.683cm}{9.095cm}}
\pgfpathlineto{\pgfqpoint{4.477cm}{9.095cm}}
\pgfpathlineto{\pgfqpoint{4.477cm}{8.301cm}}
\pgfpathlineto{\pgfqpoint{3.683cm}{8.301cm}}
\pgfpathclose
\pgfusepath{stroke}
\end{pgfscope}
\begin{pgfscope}
\definecolor{eps2pgf_color}{rgb}{1,1,0}\pgfsetstrokecolor{eps2pgf_color}\pgfsetfillcolor{eps2pgf_color}
\pgfpathmoveto{\pgfqpoint{2.095cm}{9.095cm}}
\pgfpathlineto{\pgfqpoint{2.889cm}{9.095cm}}
\pgfpathlineto{\pgfqpoint{2.889cm}{8.301cm}}
\pgfpathlineto{\pgfqpoint{2.095cm}{8.301cm}}
\pgfpathclose
\pgfseteorule\pgfusepath{fill}\pgfsetnonzerorule
\end{pgfscope}
\begin{pgfscope}
\pgfsetdash{}{0cm}
\pgfsetlinewidth{0.317mm}
\definecolor{eps2pgf_color}{gray}{0}\pgfsetstrokecolor{eps2pgf_color}\pgfsetfillcolor{eps2pgf_color}
\pgfpathmoveto{\pgfqpoint{2.095cm}{9.095cm}}
\pgfpathlineto{\pgfqpoint{2.889cm}{9.095cm}}
\pgfpathlineto{\pgfqpoint{2.889cm}{8.301cm}}
\pgfpathlineto{\pgfqpoint{2.095cm}{8.301cm}}
\pgfpathclose
\pgfusepath{stroke}
\end{pgfscope}
\begin{pgfscope}
\definecolor{eps2pgf_color}{rgb}{1,0.75,0.75}\pgfsetstrokecolor{eps2pgf_color}\pgfsetfillcolor{eps2pgf_color}
\pgfpathmoveto{\pgfqpoint{4.477cm}{9.095cm}}
\pgfpathlineto{\pgfqpoint{5.27cm}{9.095cm}}
\pgfpathlineto{\pgfqpoint{5.27cm}{8.301cm}}
\pgfpathlineto{\pgfqpoint{4.477cm}{8.301cm}}
\pgfpathclose
\pgfseteorule\pgfusepath{fill}\pgfsetnonzerorule
\end{pgfscope}
\begin{pgfscope}
\pgfsetdash{}{0cm}
\pgfsetlinewidth{0.317mm}
\definecolor{eps2pgf_color}{gray}{0}\pgfsetstrokecolor{eps2pgf_color}\pgfsetfillcolor{eps2pgf_color}
\pgfpathmoveto{\pgfqpoint{4.477cm}{9.095cm}}
\pgfpathlineto{\pgfqpoint{5.27cm}{9.095cm}}
\pgfpathlineto{\pgfqpoint{5.27cm}{8.301cm}}
\pgfpathlineto{\pgfqpoint{4.477cm}{8.301cm}}
\pgfpathclose
\pgfusepath{stroke}
\end{pgfscope}
\begin{pgfscope}
\definecolor{eps2pgf_color}{gray}{0}\pgfsetstrokecolor{eps2pgf_color}\pgfsetfillcolor{eps2pgf_color}
\pgftext[x=3.341cm,y=10.369cm,rotate=0]{\fontsize{27.1}{32.52}\selectfont{\textrm{\textbf{P}}}}
\end{pgfscope}
\begin{pgfscope}
\definecolor{eps2pgf_color}{gray}{0}\pgfsetstrokecolor{eps2pgf_color}\pgfsetfillcolor{eps2pgf_color}
\pgftext[x=3.715cm,y=9.987cm,rotate=0]{\fontsize{21.68}{26.02}\selectfont{\textrm{\textbf{0}}}}
\end{pgfscope}
\begin{pgfscope}
\definecolor{eps2pgf_color}{gray}{0}\pgfsetstrokecolor{eps2pgf_color}\pgfsetfillcolor{eps2pgf_color}
\pgftext[x=10.009cm,y=10.369cm,rotate=0]{\fontsize{27.1}{32.52}\selectfont{\textrm{\textbf{P}}}}
\end{pgfscope}
\begin{pgfscope}
\definecolor{eps2pgf_color}{gray}{0}\pgfsetstrokecolor{eps2pgf_color}\pgfsetfillcolor{eps2pgf_color}
\pgftext[x=10.385cm,y=10.15cm,rotate=0]{\fontsize{21.68}{26.02}\selectfont{\textrm{\textbf{1}}}}
\end{pgfscope}
\begin{pgfscope}
\definecolor{eps2pgf_color}{gray}{0}\pgfsetstrokecolor{eps2pgf_color}\pgfsetfillcolor{eps2pgf_color}
\pgftext[x=16.359cm,y=10.369cm,rotate=0]{\fontsize{27.1}{32.52}\selectfont{\textrm{\textbf{P}}}}
\end{pgfscope}
\begin{pgfscope}
\definecolor{eps2pgf_color}{gray}{0}\pgfsetstrokecolor{eps2pgf_color}\pgfsetfillcolor{eps2pgf_color}
\pgftext[x=16.73cm,y=9.992cm,rotate=0]{\fontsize{21.68}{26.02}\selectfont{\textrm{\textbf{2}}}}
\end{pgfscope}
\begin{pgfscope}
\definecolor{eps2pgf_color}{gray}{0}\pgfsetstrokecolor{eps2pgf_color}\pgfsetfillcolor{eps2pgf_color}
\pgftext[x=22.709cm,y=10.21cm,rotate=0]{\fontsize{27.1}{32.52}\selectfont{\textrm{\textbf{P}}}}
\end{pgfscope}
\begin{pgfscope}
\definecolor{eps2pgf_color}{gray}{0}\pgfsetstrokecolor{eps2pgf_color}\pgfsetfillcolor{eps2pgf_color}
\pgftext[x=23.076cm,y=9.986cm,rotate=0]{\fontsize{21.68}{26.02}\selectfont{\textrm{\textbf{3}}}}
\end{pgfscope}
\begin{pgfscope}
\definecolor{eps2pgf_color}{gray}{0}\pgfsetstrokecolor{eps2pgf_color}\pgfsetfillcolor{eps2pgf_color}
\pgftext[x=3.659cm,y=2.749cm,rotate=0]{\fontsize{27.1}{32.52}\selectfont{\textrm{\textbf{P}}}}
\end{pgfscope}
\begin{pgfscope}
\definecolor{eps2pgf_color}{gray}{0}\pgfsetstrokecolor{eps2pgf_color}\pgfsetfillcolor{eps2pgf_color}
\pgftext[x=4.032cm,y=2.367cm,rotate=0]{\fontsize{21.68}{26.02}\selectfont{\textrm{\textbf{0}}}}
\end{pgfscope}
\begin{pgfscope}
\definecolor{eps2pgf_color}{gray}{0}\pgfsetstrokecolor{eps2pgf_color}\pgfsetfillcolor{eps2pgf_color}
\pgftext[x=10.326cm,y=2.908cm,rotate=0]{\fontsize{27.1}{32.52}\selectfont{\textrm{\textbf{P}}}}
\end{pgfscope}
\begin{pgfscope}
\definecolor{eps2pgf_color}{gray}{0}\pgfsetstrokecolor{eps2pgf_color}\pgfsetfillcolor{eps2pgf_color}
\pgftext[x=10.702cm,y=2.689cm,rotate=0]{\fontsize{21.68}{26.02}\selectfont{\textrm{\textbf{1}}}}
\end{pgfscope}
\begin{pgfscope}
\definecolor{eps2pgf_color}{gray}{0}\pgfsetstrokecolor{eps2pgf_color}\pgfsetfillcolor{eps2pgf_color}
\pgftext[x=16.676cm,y=2.908cm,rotate=0]{\fontsize{27.1}{32.52}\selectfont{\textrm{\textbf{P}}}}
\end{pgfscope}
\begin{pgfscope}
\definecolor{eps2pgf_color}{gray}{0}\pgfsetstrokecolor{eps2pgf_color}\pgfsetfillcolor{eps2pgf_color}
\pgftext[x=17.048cm,y=2.53cm,rotate=0]{\fontsize{21.68}{26.02}\selectfont{\textrm{\textbf{2}}}}
\end{pgfscope}
\begin{pgfscope}
\definecolor{eps2pgf_color}{gray}{0}\pgfsetstrokecolor{eps2pgf_color}\pgfsetfillcolor{eps2pgf_color}
\pgftext[x=23.026cm,y=2.749cm,rotate=0]{\fontsize{27.1}{32.52}\selectfont{\textrm{\textbf{P}}}}
\end{pgfscope}
\begin{pgfscope}
\definecolor{eps2pgf_color}{gray}{0}\pgfsetstrokecolor{eps2pgf_color}\pgfsetfillcolor{eps2pgf_color}
\pgftext[x=23.394cm,y=2.525cm,rotate=0]{\fontsize{21.68}{26.02}\selectfont{\textrm{\textbf{3}}}}
\end{pgfscope}
\begin{pgfscope}
\pgfsetdash{}{0cm}
\pgfsetlinewidth{0.317mm}
\definecolor{eps2pgf_color}{gray}{0}\pgfsetstrokecolor{eps2pgf_color}\pgfsetfillcolor{eps2pgf_color}
\pgfpathmoveto{\pgfqpoint{8.445cm}{9.571cm}}
\pgfpathlineto{\pgfqpoint{12.255cm}{9.571cm}}
\pgfpathlineto{\pgfqpoint{12.255cm}{8.142cm}}
\pgfpathlineto{\pgfqpoint{8.445cm}{8.142cm}}
\pgfpathclose
\pgfusepath{stroke}
\end{pgfscope}
\begin{pgfscope}
\pgfsetdash{}{0cm}
\pgfsetlinewidth{0.317mm}
\definecolor{eps2pgf_color}{gray}{0}\pgfsetstrokecolor{eps2pgf_color}\pgfsetfillcolor{eps2pgf_color}
\pgfpathmoveto{\pgfqpoint{14.795cm}{9.571cm}}
\pgfpathlineto{\pgfqpoint{18.605cm}{9.571cm}}
\pgfpathlineto{\pgfqpoint{18.605cm}{8.142cm}}
\pgfpathlineto{\pgfqpoint{14.795cm}{8.142cm}}
\pgfpathclose
\pgfusepath{stroke}
\end{pgfscope}
\begin{pgfscope}
\pgfsetdash{}{0cm}
\pgfsetlinewidth{0.317mm}
\definecolor{eps2pgf_color}{gray}{0}\pgfsetstrokecolor{eps2pgf_color}\pgfsetfillcolor{eps2pgf_color}
\pgfpathmoveto{\pgfqpoint{21.145cm}{9.571cm}}
\pgfpathlineto{\pgfqpoint{24.955cm}{9.571cm}}
\pgfpathlineto{\pgfqpoint{24.955cm}{8.142cm}}
\pgfpathlineto{\pgfqpoint{21.145cm}{8.142cm}}
\pgfpathclose
\pgfusepath{stroke}
\end{pgfscope}
\begin{pgfscope}
\pgfsetdash{}{0cm}
\pgfsetlinewidth{0.317mm}
\definecolor{eps2pgf_color}{gray}{0}\pgfsetstrokecolor{eps2pgf_color}\pgfsetfillcolor{eps2pgf_color}
\pgfpathmoveto{\pgfqpoint{8.763cm}{9.253cm}}
\pgfpathlineto{\pgfqpoint{9.557cm}{9.253cm}}
\pgfpathlineto{\pgfqpoint{9.557cm}{8.46cm}}
\pgfpathlineto{\pgfqpoint{8.763cm}{8.46cm}}
\pgfpathclose
\pgfusepath{stroke}
\end{pgfscope}
\begin{pgfscope}
\pgfsetdash{}{0cm}
\pgfsetlinewidth{0.317mm}
\definecolor{eps2pgf_color}{gray}{0}\pgfsetstrokecolor{eps2pgf_color}\pgfsetfillcolor{eps2pgf_color}
\pgfpathmoveto{\pgfqpoint{9.557cm}{9.253cm}}
\pgfpathlineto{\pgfqpoint{10.35cm}{9.253cm}}
\pgfpathlineto{\pgfqpoint{10.35cm}{8.46cm}}
\pgfpathlineto{\pgfqpoint{9.557cm}{8.46cm}}
\pgfpathclose
\pgfusepath{stroke}
\end{pgfscope}
\begin{pgfscope}
\pgfsetdash{}{0cm}
\pgfsetlinewidth{0.317mm}
\definecolor{eps2pgf_color}{gray}{0}\pgfsetstrokecolor{eps2pgf_color}\pgfsetfillcolor{eps2pgf_color}
\pgfpathmoveto{\pgfqpoint{10.35cm}{9.253cm}}
\pgfpathlineto{\pgfqpoint{11.144cm}{9.253cm}}
\pgfpathlineto{\pgfqpoint{11.144cm}{8.46cm}}
\pgfpathlineto{\pgfqpoint{10.35cm}{8.46cm}}
\pgfpathclose
\pgfusepath{stroke}
\end{pgfscope}
\begin{pgfscope}
\pgfsetdash{}{0cm}
\pgfsetlinewidth{0.317mm}
\definecolor{eps2pgf_color}{gray}{0}\pgfsetstrokecolor{eps2pgf_color}\pgfsetfillcolor{eps2pgf_color}
\pgfpathmoveto{\pgfqpoint{11.144cm}{9.253cm}}
\pgfpathlineto{\pgfqpoint{11.938cm}{9.253cm}}
\pgfpathlineto{\pgfqpoint{11.938cm}{8.46cm}}
\pgfpathlineto{\pgfqpoint{11.144cm}{8.46cm}}
\pgfpathclose
\pgfusepath{stroke}
\end{pgfscope}
\begin{pgfscope}
\pgfsetdash{}{0cm}
\pgfsetlinewidth{0.317mm}
\definecolor{eps2pgf_color}{gray}{0}\pgfsetstrokecolor{eps2pgf_color}\pgfsetfillcolor{eps2pgf_color}
\pgfpathmoveto{\pgfqpoint{15.113cm}{9.253cm}}
\pgfpathlineto{\pgfqpoint{15.907cm}{9.253cm}}
\pgfpathlineto{\pgfqpoint{15.907cm}{8.46cm}}
\pgfpathlineto{\pgfqpoint{15.113cm}{8.46cm}}
\pgfpathclose
\pgfusepath{stroke}
\end{pgfscope}
\begin{pgfscope}
\pgfsetdash{}{0cm}
\pgfsetlinewidth{0.317mm}
\definecolor{eps2pgf_color}{gray}{0}\pgfsetstrokecolor{eps2pgf_color}\pgfsetfillcolor{eps2pgf_color}
\pgfpathmoveto{\pgfqpoint{15.907cm}{9.253cm}}
\pgfpathlineto{\pgfqpoint{16.7cm}{9.253cm}}
\pgfpathlineto{\pgfqpoint{16.7cm}{8.46cm}}
\pgfpathlineto{\pgfqpoint{15.907cm}{8.46cm}}
\pgfpathclose
\pgfusepath{stroke}
\end{pgfscope}
\begin{pgfscope}
\pgfsetdash{}{0cm}
\pgfsetlinewidth{0.317mm}
\definecolor{eps2pgf_color}{gray}{0}\pgfsetstrokecolor{eps2pgf_color}\pgfsetfillcolor{eps2pgf_color}
\pgfpathmoveto{\pgfqpoint{16.7cm}{9.253cm}}
\pgfpathlineto{\pgfqpoint{17.494cm}{9.253cm}}
\pgfpathlineto{\pgfqpoint{17.494cm}{8.46cm}}
\pgfpathlineto{\pgfqpoint{16.7cm}{8.46cm}}
\pgfpathclose
\pgfusepath{stroke}
\end{pgfscope}
\begin{pgfscope}
\pgfsetdash{}{0cm}
\pgfsetlinewidth{0.317mm}
\definecolor{eps2pgf_color}{gray}{0}\pgfsetstrokecolor{eps2pgf_color}\pgfsetfillcolor{eps2pgf_color}
\pgfpathmoveto{\pgfqpoint{17.494cm}{9.253cm}}
\pgfpathlineto{\pgfqpoint{18.288cm}{9.253cm}}
\pgfpathlineto{\pgfqpoint{18.288cm}{8.46cm}}
\pgfpathlineto{\pgfqpoint{17.494cm}{8.46cm}}
\pgfpathclose
\pgfusepath{stroke}
\end{pgfscope}
\begin{pgfscope}
\pgfsetdash{}{0cm}
\pgfsetlinewidth{0.317mm}
\definecolor{eps2pgf_color}{gray}{0}\pgfsetstrokecolor{eps2pgf_color}\pgfsetfillcolor{eps2pgf_color}
\pgfpathmoveto{\pgfqpoint{21.463cm}{9.253cm}}
\pgfpathlineto{\pgfqpoint{22.257cm}{9.253cm}}
\pgfpathlineto{\pgfqpoint{22.257cm}{8.46cm}}
\pgfpathlineto{\pgfqpoint{21.463cm}{8.46cm}}
\pgfpathclose
\pgfusepath{stroke}
\end{pgfscope}
\begin{pgfscope}
\pgfsetdash{}{0cm}
\pgfsetlinewidth{0.317mm}
\definecolor{eps2pgf_color}{gray}{0}\pgfsetstrokecolor{eps2pgf_color}\pgfsetfillcolor{eps2pgf_color}
\pgfpathmoveto{\pgfqpoint{22.257cm}{9.253cm}}
\pgfpathlineto{\pgfqpoint{23.05cm}{9.253cm}}
\pgfpathlineto{\pgfqpoint{23.05cm}{8.46cm}}
\pgfpathlineto{\pgfqpoint{22.257cm}{8.46cm}}
\pgfpathclose
\pgfusepath{stroke}
\end{pgfscope}
\begin{pgfscope}
\pgfsetdash{}{0cm}
\pgfsetlinewidth{0.317mm}
\definecolor{eps2pgf_color}{gray}{0}\pgfsetstrokecolor{eps2pgf_color}\pgfsetfillcolor{eps2pgf_color}
\pgfpathmoveto{\pgfqpoint{23.05cm}{9.253cm}}
\pgfpathlineto{\pgfqpoint{23.844cm}{9.253cm}}
\pgfpathlineto{\pgfqpoint{23.844cm}{8.46cm}}
\pgfpathlineto{\pgfqpoint{23.05cm}{8.46cm}}
\pgfpathclose
\pgfusepath{stroke}
\end{pgfscope}
\begin{pgfscope}
\pgfsetdash{}{0cm}
\pgfsetlinewidth{0.317mm}
\definecolor{eps2pgf_color}{gray}{0}\pgfsetstrokecolor{eps2pgf_color}\pgfsetfillcolor{eps2pgf_color}
\pgfpathmoveto{\pgfqpoint{23.844cm}{9.253cm}}
\pgfpathlineto{\pgfqpoint{24.638cm}{9.253cm}}
\pgfpathlineto{\pgfqpoint{24.638cm}{8.46cm}}
\pgfpathlineto{\pgfqpoint{23.844cm}{8.46cm}}
\pgfpathclose
\pgfusepath{stroke}
\end{pgfscope}
\begin{pgfscope}
\pgfpathmoveto{\pgfqpoint{0cm}{10.76cm}}
\pgfpathlineto{\pgfqpoint{0cm}{0cm}}
\pgfpathlineto{\pgfqpoint{25.329cm}{0cm}}
\pgfpathlineto{\pgfqpoint{25.329cm}{10.76cm}}
\pgfpathclose
\pgfpathmoveto{\pgfqpoint{11.834cm}{3.166cm}}
\pgfpathlineto{\pgfqpoint{11.726cm}{3.166cm}}
\pgfpathlineto{\pgfqpoint{11.525cm}{4.442cm}}
\pgfpathlineto{\pgfqpoint{12.033cm}{4.442cm}}
\pgfpathclose
\pgfseteorule\pgfusepath{clip}\pgfsetnonzerorule
\begin{pgfscope}
\pgfsetdash{}{0cm}
\pgfsetlinewidth{0.952mm}
\definecolor{eps2pgf_color}{gray}{0}\pgfsetstrokecolor{eps2pgf_color}\pgfsetfillcolor{eps2pgf_color}
\pgfpathmoveto{\pgfqpoint{11.779cm}{7.031cm}}
\pgfpathlineto{\pgfqpoint{11.779cm}{3.221cm}}
\pgfusepath{stroke}
\end{pgfscope}
\end{pgfscope}
\begin{pgfscope}
\definecolor{eps2pgf_color}{gray}{0}\pgfsetstrokecolor{eps2pgf_color}\pgfsetfillcolor{eps2pgf_color}
\pgfpathmoveto{\pgfqpoint{11.525cm}{4.442cm}}
\pgfpathlineto{\pgfqpoint{11.779cm}{3.426cm}}
\pgfpathlineto{\pgfqpoint{12.033cm}{4.442cm}}
\pgfpathlineto{\pgfqpoint{11.525cm}{4.442cm}}
\pgfpathclose
\pgfseteorule\pgfusepath{fill}\pgfsetnonzerorule
\end{pgfscope}
\pgfsetdash{}{0cm}
\pgfsetlinewidth{0.952mm}
\definecolor{eps2pgf_color}{gray}{0}\pgfsetstrokecolor{eps2pgf_color}\pgfsetfillcolor{eps2pgf_color}
\pgfpathmoveto{\pgfqpoint{11.525cm}{4.442cm}}
\pgfpathlineto{\pgfqpoint{11.779cm}{3.426cm}}
\pgfpathlineto{\pgfqpoint{12.033cm}{4.442cm}}
\pgfpathlineto{\pgfqpoint{11.525cm}{4.442cm}}
\pgfpathclose
\pgfusepath{stroke}
\begin{pgfscope}
\pgfsetdash{}{0cm}
\pgfsetlinewidth{0.317mm}
\pgfpathmoveto{\pgfqpoint{1.937cm}{1.475cm}}
\pgfpathlineto{\pgfqpoint{5.747cm}{1.475cm}}
\pgfpathlineto{\pgfqpoint{5.747cm}{0.046cm}}
\pgfpathlineto{\pgfqpoint{1.937cm}{0.046cm}}
\pgfpathclose
\pgfusepath{stroke}
\end{pgfscope}
\begin{pgfscope}
\pgfsetdash{}{0cm}
\pgfsetlinewidth{0.317mm}
\pgfpathmoveto{\pgfqpoint{8.604cm}{1.633cm}}
\pgfpathlineto{\pgfqpoint{12.414cm}{1.633cm}}
\pgfpathlineto{\pgfqpoint{12.414cm}{0.205cm}}
\pgfpathlineto{\pgfqpoint{8.604cm}{0.205cm}}
\pgfpathclose
\pgfusepath{stroke}
\end{pgfscope}
\begin{pgfscope}
\pgfsetdash{}{0cm}
\pgfsetlinewidth{0.317mm}
\pgfpathmoveto{\pgfqpoint{14.954cm}{1.633cm}}
\pgfpathlineto{\pgfqpoint{18.764cm}{1.633cm}}
\pgfpathlineto{\pgfqpoint{18.764cm}{0.205cm}}
\pgfpathlineto{\pgfqpoint{14.954cm}{0.205cm}}
\pgfpathclose
\pgfusepath{stroke}
\end{pgfscope}
\begin{pgfscope}
\pgfsetdash{}{0cm}
\pgfsetlinewidth{0.317mm}
\pgfpathmoveto{\pgfqpoint{21.463cm}{1.792cm}}
\pgfpathlineto{\pgfqpoint{25.273cm}{1.792cm}}
\pgfpathlineto{\pgfqpoint{25.273cm}{0.363cm}}
\pgfpathlineto{\pgfqpoint{21.463cm}{0.363cm}}
\pgfpathclose
\pgfusepath{stroke}
\end{pgfscope}
\begin{pgfscope}
\pgfsetdash{}{0cm}
\pgfsetlinewidth{0.317mm}
\pgfpathmoveto{\pgfqpoint{4.635cm}{1.157cm}}
\pgfpathlineto{\pgfqpoint{5.429cm}{1.157cm}}
\pgfpathlineto{\pgfqpoint{5.429cm}{0.363cm}}
\pgfpathlineto{\pgfqpoint{4.635cm}{0.363cm}}
\pgfpathclose
\pgfusepath{stroke}
\end{pgfscope}
\begin{pgfscope}
\pgfsetdash{}{0cm}
\pgfsetlinewidth{0.317mm}
\pgfpathmoveto{\pgfqpoint{3.842cm}{1.157cm}}
\pgfpathlineto{\pgfqpoint{4.635cm}{1.157cm}}
\pgfpathlineto{\pgfqpoint{4.635cm}{0.363cm}}
\pgfpathlineto{\pgfqpoint{3.842cm}{0.363cm}}
\pgfpathclose
\pgfusepath{stroke}
\end{pgfscope}
\begin{pgfscope}
\pgfsetdash{}{0cm}
\pgfsetlinewidth{0.317mm}
\pgfpathmoveto{\pgfqpoint{3.048cm}{1.157cm}}
\pgfpathlineto{\pgfqpoint{3.842cm}{1.157cm}}
\pgfpathlineto{\pgfqpoint{3.842cm}{0.363cm}}
\pgfpathlineto{\pgfqpoint{3.048cm}{0.363cm}}
\pgfpathclose
\pgfusepath{stroke}
\end{pgfscope}
\begin{pgfscope}
\pgfsetdash{}{0cm}
\pgfsetlinewidth{0.317mm}
\pgfpathmoveto{\pgfqpoint{9.715cm}{1.316cm}}
\pgfpathlineto{\pgfqpoint{10.509cm}{1.316cm}}
\pgfpathlineto{\pgfqpoint{10.509cm}{0.522cm}}
\pgfpathlineto{\pgfqpoint{9.715cm}{0.522cm}}
\pgfpathclose
\pgfusepath{stroke}
\end{pgfscope}
\begin{pgfscope}
\pgfsetdash{}{0cm}
\pgfsetlinewidth{0.317mm}
\pgfpathmoveto{\pgfqpoint{10.509cm}{1.316cm}}
\pgfpathlineto{\pgfqpoint{11.303cm}{1.316cm}}
\pgfpathlineto{\pgfqpoint{11.303cm}{0.522cm}}
\pgfpathlineto{\pgfqpoint{10.509cm}{0.522cm}}
\pgfpathclose
\pgfusepath{stroke}
\end{pgfscope}
\begin{pgfscope}
\pgfsetdash{}{0cm}
\pgfsetlinewidth{0.317mm}
\pgfpathmoveto{\pgfqpoint{11.303cm}{1.316cm}}
\pgfpathlineto{\pgfqpoint{12.097cm}{1.316cm}}
\pgfpathlineto{\pgfqpoint{12.097cm}{0.522cm}}
\pgfpathlineto{\pgfqpoint{11.303cm}{0.522cm}}
\pgfpathclose
\pgfusepath{stroke}
\end{pgfscope}
\begin{pgfscope}
\pgfsetdash{}{0cm}
\pgfsetlinewidth{0.317mm}
\pgfpathmoveto{\pgfqpoint{16.065cm}{1.316cm}}
\pgfpathlineto{\pgfqpoint{16.859cm}{1.316cm}}
\pgfpathlineto{\pgfqpoint{16.859cm}{0.522cm}}
\pgfpathlineto{\pgfqpoint{16.065cm}{0.522cm}}
\pgfpathclose
\pgfusepath{stroke}
\end{pgfscope}
\begin{pgfscope}
\pgfsetdash{}{0cm}
\pgfsetlinewidth{0.317mm}
\pgfpathmoveto{\pgfqpoint{16.859cm}{1.316cm}}
\pgfpathlineto{\pgfqpoint{17.653cm}{1.316cm}}
\pgfpathlineto{\pgfqpoint{17.653cm}{0.522cm}}
\pgfpathlineto{\pgfqpoint{16.859cm}{0.522cm}}
\pgfpathclose
\pgfusepath{stroke}
\end{pgfscope}
\begin{pgfscope}
\pgfsetdash{}{0cm}
\pgfsetlinewidth{0.317mm}
\pgfpathmoveto{\pgfqpoint{17.653cm}{1.316cm}}
\pgfpathlineto{\pgfqpoint{18.447cm}{1.316cm}}
\pgfpathlineto{\pgfqpoint{18.447cm}{0.522cm}}
\pgfpathlineto{\pgfqpoint{17.653cm}{0.522cm}}
\pgfpathclose
\pgfusepath{stroke}
\end{pgfscope}
\begin{pgfscope}
\pgfsetdash{}{0cm}
\pgfsetlinewidth{0.317mm}
\pgfpathmoveto{\pgfqpoint{22.574cm}{1.475cm}}
\pgfpathlineto{\pgfqpoint{23.368cm}{1.475cm}}
\pgfpathlineto{\pgfqpoint{23.368cm}{0.681cm}}
\pgfpathlineto{\pgfqpoint{22.574cm}{0.681cm}}
\pgfpathclose
\pgfusepath{stroke}
\end{pgfscope}
\begin{pgfscope}
\pgfsetdash{}{0cm}
\pgfsetlinewidth{0.317mm}
\pgfpathmoveto{\pgfqpoint{23.368cm}{1.475cm}}
\pgfpathlineto{\pgfqpoint{24.162cm}{1.475cm}}
\pgfpathlineto{\pgfqpoint{24.162cm}{0.681cm}}
\pgfpathlineto{\pgfqpoint{23.368cm}{0.681cm}}
\pgfpathclose
\pgfusepath{stroke}
\end{pgfscope}
\begin{pgfscope}
\pgfsetdash{}{0cm}
\pgfsetlinewidth{0.317mm}
\pgfpathmoveto{\pgfqpoint{24.162cm}{1.475cm}}
\pgfpathlineto{\pgfqpoint{24.955cm}{1.475cm}}
\pgfpathlineto{\pgfqpoint{24.955cm}{0.681cm}}
\pgfpathlineto{\pgfqpoint{24.162cm}{0.681cm}}
\pgfpathclose
\pgfusepath{stroke}
\end{pgfscope}
\begin{pgfscope}
\pgfsetdash{}{0cm}
\pgfsetlinewidth{0.317mm}
\pgfpathmoveto{\pgfqpoint{1.778cm}{9.412cm}}
\pgfpathlineto{\pgfqpoint{5.588cm}{9.412cm}}
\pgfpathlineto{\pgfqpoint{5.588cm}{7.983cm}}
\pgfpathlineto{\pgfqpoint{1.778cm}{7.983cm}}
\pgfpathclose
\pgfusepath{stroke}
\end{pgfscope}
\begin{pgfscope}
\pgftext[x=7.567cm,y=9.647cm,rotate=0]{\fontsize{27.1}{32.52}\selectfont{\textrm{\textbf{\textit{buf}}}}}
\end{pgfscope}
\begin{pgfscope}
\pgftext[x=13.917cm,y=9.647cm,rotate=0]{\fontsize{27.1}{32.52}\selectfont{\textrm{\textbf{\textit{buf}}}}}
\end{pgfscope}
\begin{pgfscope}
\pgftext[x=20.267cm,y=9.647cm,rotate=0]{\fontsize{27.1}{32.52}\selectfont{\textrm{\textbf{\textit{buf}}}}}
\end{pgfscope}
\begin{pgfscope}
\pgftext[x=1.058cm,y=1.551cm,rotate=0]{\fontsize{27.1}{32.52}\selectfont{\textrm{\textbf{\textit{buf}}}}}
\end{pgfscope}
\begin{pgfscope}
\pgftext[x=7.725cm,y=1.71cm,rotate=0]{\fontsize{27.1}{32.52}\selectfont{\textrm{\textbf{\textit{buf}}}}}
\end{pgfscope}
\begin{pgfscope}
\pgftext[x=14.075cm,y=1.71cm,rotate=0]{\fontsize{27.1}{32.52}\selectfont{\textrm{\textbf{\textit{buf}}}}}
\end{pgfscope}
\begin{pgfscope}
\pgftext[x=20.584cm,y=1.869cm,rotate=0]{\fontsize{27.1}{32.52}\selectfont{\textrm{\textbf{\textit{buf}}}}}
\end{pgfscope}
\begin{pgfscope}
\pgftext[x=0.899cm,y=9.489cm,rotate=0]{\fontsize{27.1}{32.52}\selectfont{\textrm{\textbf{\textit{buf}}}}}
\end{pgfscope}
\begin{pgfscope}
\definecolor{eps2pgf_color}{rgb}{1,0,0}\pgfsetstrokecolor{eps2pgf_color}\pgfsetfillcolor{eps2pgf_color}
\pgftext[x=0.93cm,y=10.374cm,rotate=0]{\fontsize{27.1}{32.52}\selectfont{\textrm{\textbf{\textit{send}}}}}
\end{pgfscope}
\begin{pgfscope}
\definecolor{eps2pgf_color}{rgb}{1,0,0}\pgfsetstrokecolor{eps2pgf_color}\pgfsetfillcolor{eps2pgf_color}
\pgftext[x=0.98cm,y=2.323cm,rotate=0]{\fontsize{27.1}{32.52}\selectfont{\textrm{\textbf{\textit{recv}}}}}
\end{pgfscope}
\begin{pgfscope}
\definecolor{eps2pgf_color}{rgb}{1,0,0}\pgfsetstrokecolor{eps2pgf_color}\pgfsetfillcolor{eps2pgf_color}
\pgftext[x=7.28cm,y=10.374cm,rotate=0]{\fontsize{27.1}{32.52}\selectfont{\textrm{\textbf{\textit{send}}}}}
\end{pgfscope}
\begin{pgfscope}
\definecolor{eps2pgf_color}{rgb}{1,0,0}\pgfsetstrokecolor{eps2pgf_color}\pgfsetfillcolor{eps2pgf_color}
\pgftext[x=13.63cm,y=10.374cm,rotate=0]{\fontsize{27.1}{32.52}\selectfont{\textrm{\textbf{\textit{send}}}}}
\end{pgfscope}
\begin{pgfscope}
\definecolor{eps2pgf_color}{rgb}{1,0,0}\pgfsetstrokecolor{eps2pgf_color}\pgfsetfillcolor{eps2pgf_color}
\pgftext[x=20.139cm,y=10.374cm,rotate=0]{\fontsize{27.1}{32.52}\selectfont{\textrm{\textbf{\textit{send}}}}}
\end{pgfscope}
\begin{pgfscope}
\definecolor{eps2pgf_color}{rgb}{1,0,0}\pgfsetstrokecolor{eps2pgf_color}\pgfsetfillcolor{eps2pgf_color}
\pgftext[x=7.647cm,y=2.323cm,rotate=0]{\fontsize{27.1}{32.52}\selectfont{\textrm{\textbf{\textit{recv}}}}}
\end{pgfscope}
\begin{pgfscope}
\definecolor{eps2pgf_color}{rgb}{1,0,0}\pgfsetstrokecolor{eps2pgf_color}\pgfsetfillcolor{eps2pgf_color}
\pgftext[x=13.997cm,y=2.323cm,rotate=0]{\fontsize{27.1}{32.52}\selectfont{\textrm{\textbf{\textit{recv}}}}}
\end{pgfscope}
\begin{pgfscope}
\definecolor{eps2pgf_color}{rgb}{1,0,0}\pgfsetstrokecolor{eps2pgf_color}\pgfsetfillcolor{eps2pgf_color}
\pgftext[x=20.506cm,y=2.482cm,rotate=0]{\fontsize{27.1}{32.52}\selectfont{\textrm{\textbf{\textit{recv}}}}}
\end{pgfscope}
\begin{pgfscope}
\definecolor{eps2pgf_color}{rgb}{1,0,0}\pgfsetstrokecolor{eps2pgf_color}\pgfsetfillcolor{eps2pgf_color}
\pgftext[x=10.84cm,y=5.212cm,rotate=0]{\fontsize{27.1}{32.52}\selectfont{\textrm{\textbf{MPI{\_}Scatter(sendbuf, ..., recvbuf, .., 0, ... );}}}}
\end{pgfscope}
\end{pgfscope}
\end{pgfscope}
\end{pgfpicture}

    }
    \caption{Visualización de MPI\_Scatter \cite{cheung_mpi}}
    \label{fig:mpi_scatter}
  \end{figure}

  \item \textbf{MPI\_Gather}: Recoge varios fragmentos de un mensaje y los combina. Es la operación inversa al scatter.

  \begin{figure}[H]
    \vspace*{0.5cm}
    \centering
    \resizebox {0.8\textwidth} {!} {
    \input{pgf/MPI_Gather.pgf}
    }
    \caption{Visualización de MPI\_Gather \cite{cheung_mpi}}
    \label{fig:mpi_gather}
  \end{figure}

  \item \textbf{MPI\_Reduce}: Recoge al igual que en el scatter los fragmentos del mensaje, pero en lugar de almacenarlos como tal, aplica una operación sobre ellos.

  \begin{figure}[H]
    \vspace*{0.5cm}
    \centering
    \resizebox {0.8\textwidth} {!} {
    % Created by Eps2pgf 0.7.0 (build on 2008-08-24) on Sun Aug 22 14:18:31 CEST 2021
\begin{pgfpicture}
\pgfpathmoveto{\pgfqpoint{0cm}{0cm}}
\pgfpathlineto{\pgfqpoint{25.329cm}{0cm}}
\pgfpathlineto{\pgfqpoint{25.329cm}{10.407cm}}
\pgfpathlineto{\pgfqpoint{0cm}{10.407cm}}
\pgfpathclose
\pgfusepath{clip}
\begin{pgfscope}
\pgfpathmoveto{\pgfqpoint{0cm}{10.407cm}}
\pgfpathlineto{\pgfqpoint{0cm}{0cm}}
\pgfpathlineto{\pgfqpoint{25.329cm}{0cm}}
\pgfpathlineto{\pgfqpoint{25.329cm}{10.407cm}}
\pgfpathclose
\pgfusepath{clip}
\begin{pgfscope}
\begin{pgfscope}
\definecolor{eps2pgf_color}{rgb}{1,1,0}\pgfsetstrokecolor{eps2pgf_color}\pgfsetfillcolor{eps2pgf_color}
\pgfpathmoveto{\pgfqpoint{2.254cm}{5.761cm}}
\pgfpathlineto{\pgfqpoint{25.273cm}{5.761cm}}
\pgfpathlineto{\pgfqpoint{25.273cm}{4.015cm}}
\pgfpathlineto{\pgfqpoint{2.254cm}{4.015cm}}
\pgfpathclose
\pgfseteorule\pgfusepath{fill}\pgfsetnonzerorule
\end{pgfscope}
\begin{pgfscope}
\pgfsetdash{}{0cm}
\pgfsetlinewidth{0.317mm}
\definecolor{eps2pgf_color}{rgb}{1,1,0}\pgfsetstrokecolor{eps2pgf_color}\pgfsetfillcolor{eps2pgf_color}
\pgfpathmoveto{\pgfqpoint{2.254cm}{5.761cm}}
\pgfpathlineto{\pgfqpoint{25.273cm}{5.761cm}}
\pgfpathlineto{\pgfqpoint{25.273cm}{4.015cm}}
\pgfpathlineto{\pgfqpoint{2.254cm}{4.015cm}}
\pgfpathclose
\pgfusepath{stroke}
\end{pgfscope}
\begin{pgfscope}
\definecolor{eps2pgf_color}{rgb}{1,1,0}\pgfsetstrokecolor{eps2pgf_color}\pgfsetfillcolor{eps2pgf_color}
\pgfpathmoveto{\pgfqpoint{2.095cm}{8.301cm}}
\pgfpathlineto{\pgfqpoint{2.889cm}{8.301cm}}
\pgfpathlineto{\pgfqpoint{2.889cm}{7.507cm}}
\pgfpathlineto{\pgfqpoint{2.095cm}{7.507cm}}
\pgfpathclose
\pgfseteorule\pgfusepath{fill}\pgfsetnonzerorule
\end{pgfscope}
\begin{pgfscope}
\pgfsetdash{}{0cm}
\pgfsetlinewidth{0.317mm}
\definecolor{eps2pgf_color}{gray}{0}\pgfsetstrokecolor{eps2pgf_color}\pgfsetfillcolor{eps2pgf_color}
\pgfpathmoveto{\pgfqpoint{2.095cm}{8.301cm}}
\pgfpathlineto{\pgfqpoint{2.889cm}{8.301cm}}
\pgfpathlineto{\pgfqpoint{2.889cm}{7.507cm}}
\pgfpathlineto{\pgfqpoint{2.095cm}{7.507cm}}
\pgfpathclose
\pgfusepath{stroke}
\end{pgfscope}
\begin{pgfscope}
\definecolor{eps2pgf_color}{rgb}{0.53,0.81,1}\pgfsetstrokecolor{eps2pgf_color}\pgfsetfillcolor{eps2pgf_color}
\pgfpathmoveto{\pgfqpoint{8.763cm}{8.46cm}}
\pgfpathlineto{\pgfqpoint{9.557cm}{8.46cm}}
\pgfpathlineto{\pgfqpoint{9.557cm}{7.666cm}}
\pgfpathlineto{\pgfqpoint{8.763cm}{7.666cm}}
\pgfpathclose
\pgfseteorule\pgfusepath{fill}\pgfsetnonzerorule
\end{pgfscope}
\begin{pgfscope}
\pgfsetdash{}{0cm}
\pgfsetlinewidth{0.317mm}
\definecolor{eps2pgf_color}{gray}{0}\pgfsetstrokecolor{eps2pgf_color}\pgfsetfillcolor{eps2pgf_color}
\pgfpathmoveto{\pgfqpoint{8.763cm}{8.46cm}}
\pgfpathlineto{\pgfqpoint{9.557cm}{8.46cm}}
\pgfpathlineto{\pgfqpoint{9.557cm}{7.666cm}}
\pgfpathlineto{\pgfqpoint{8.763cm}{7.666cm}}
\pgfpathclose
\pgfusepath{stroke}
\end{pgfscope}
\begin{pgfscope}
\definecolor{eps2pgf_color}{rgb}{0,1,0}\pgfsetstrokecolor{eps2pgf_color}\pgfsetfillcolor{eps2pgf_color}
\pgfpathmoveto{\pgfqpoint{15.113cm}{8.46cm}}
\pgfpathlineto{\pgfqpoint{15.907cm}{8.46cm}}
\pgfpathlineto{\pgfqpoint{15.907cm}{7.666cm}}
\pgfpathlineto{\pgfqpoint{15.113cm}{7.666cm}}
\pgfpathclose
\pgfseteorule\pgfusepath{fill}\pgfsetnonzerorule
\end{pgfscope}
\begin{pgfscope}
\pgfsetdash{}{0cm}
\pgfsetlinewidth{0.317mm}
\definecolor{eps2pgf_color}{gray}{0}\pgfsetstrokecolor{eps2pgf_color}\pgfsetfillcolor{eps2pgf_color}
\pgfpathmoveto{\pgfqpoint{15.113cm}{8.46cm}}
\pgfpathlineto{\pgfqpoint{15.907cm}{8.46cm}}
\pgfpathlineto{\pgfqpoint{15.907cm}{7.666cm}}
\pgfpathlineto{\pgfqpoint{15.113cm}{7.666cm}}
\pgfpathclose
\pgfusepath{stroke}
\end{pgfscope}
\begin{pgfscope}
\definecolor{eps2pgf_color}{rgb}{1,0.75,0.75}\pgfsetstrokecolor{eps2pgf_color}\pgfsetfillcolor{eps2pgf_color}
\pgfpathmoveto{\pgfqpoint{21.622cm}{8.618cm}}
\pgfpathlineto{\pgfqpoint{22.415cm}{8.618cm}}
\pgfpathlineto{\pgfqpoint{22.415cm}{7.825cm}}
\pgfpathlineto{\pgfqpoint{21.622cm}{7.825cm}}
\pgfpathclose
\pgfseteorule\pgfusepath{fill}\pgfsetnonzerorule
\end{pgfscope}
\begin{pgfscope}
\pgfsetdash{}{0cm}
\pgfsetlinewidth{0.317mm}
\definecolor{eps2pgf_color}{gray}{0}\pgfsetstrokecolor{eps2pgf_color}\pgfsetfillcolor{eps2pgf_color}
\pgfpathmoveto{\pgfqpoint{21.622cm}{8.618cm}}
\pgfpathlineto{\pgfqpoint{22.415cm}{8.618cm}}
\pgfpathlineto{\pgfqpoint{22.415cm}{7.825cm}}
\pgfpathlineto{\pgfqpoint{21.622cm}{7.825cm}}
\pgfpathclose
\pgfusepath{stroke}
\end{pgfscope}
\begin{pgfscope}
\definecolor{eps2pgf_color}{rgb}{1,0,0}\pgfsetstrokecolor{eps2pgf_color}\pgfsetfillcolor{eps2pgf_color}
\pgfpathmoveto{\pgfqpoint{2.413cm}{1.157cm}}
\pgfpathlineto{\pgfqpoint{3.207cm}{1.157cm}}
\pgfpathlineto{\pgfqpoint{3.207cm}{0.363cm}}
\pgfpathlineto{\pgfqpoint{2.413cm}{0.363cm}}
\pgfpathclose
\pgfseteorule\pgfusepath{fill}\pgfsetnonzerorule
\end{pgfscope}
\begin{pgfscope}
\pgfsetdash{}{0cm}
\pgfsetlinewidth{0.317mm}
\definecolor{eps2pgf_color}{gray}{0}\pgfsetstrokecolor{eps2pgf_color}\pgfsetfillcolor{eps2pgf_color}
\pgfpathmoveto{\pgfqpoint{2.413cm}{1.157cm}}
\pgfpathlineto{\pgfqpoint{3.207cm}{1.157cm}}
\pgfpathlineto{\pgfqpoint{3.207cm}{0.363cm}}
\pgfpathlineto{\pgfqpoint{2.413cm}{0.363cm}}
\pgfpathclose
\pgfusepath{stroke}
\end{pgfscope}
\begin{pgfscope}
\definecolor{eps2pgf_color}{rgb}{0,1,1}\pgfsetstrokecolor{eps2pgf_color}\pgfsetfillcolor{eps2pgf_color}
\pgfpathmoveto{\pgfqpoint{17.812cm}{5.602cm}}
\pgfpathlineto{\pgfqpoint{21.78cm}{5.602cm}}
\pgfpathlineto{\pgfqpoint{21.78cm}{4.332cm}}
\pgfpathlineto{\pgfqpoint{17.812cm}{4.332cm}}
\pgfpathclose
\pgfseteorule\pgfusepath{fill}\pgfsetnonzerorule
\end{pgfscope}
\begin{pgfscope}
\pgfsetdash{}{0cm}
\pgfsetlinewidth{0.317mm}
\definecolor{eps2pgf_color}{rgb}{0,1,1}\pgfsetstrokecolor{eps2pgf_color}\pgfsetfillcolor{eps2pgf_color}
\pgfpathmoveto{\pgfqpoint{17.812cm}{5.602cm}}
\pgfpathlineto{\pgfqpoint{21.78cm}{5.602cm}}
\pgfpathlineto{\pgfqpoint{21.78cm}{4.332cm}}
\pgfpathlineto{\pgfqpoint{17.812cm}{4.332cm}}
\pgfpathclose
\pgfusepath{stroke}
\end{pgfscope}
\begin{pgfscope}
\definecolor{eps2pgf_color}{gray}{0}\pgfsetstrokecolor{eps2pgf_color}\pgfsetfillcolor{eps2pgf_color}
\pgftext[x=3.5cm,y=9.893cm,rotate=0]{\fontsize{27.1}{32.52}\selectfont{\textrm{\textbf{P}}}}
\end{pgfscope}
\begin{pgfscope}
\definecolor{eps2pgf_color}{gray}{0}\pgfsetstrokecolor{eps2pgf_color}\pgfsetfillcolor{eps2pgf_color}
\pgftext[x=3.873cm,y=9.511cm,rotate=0]{\fontsize{21.68}{26.02}\selectfont{\textrm{\textbf{0}}}}
\end{pgfscope}
\begin{pgfscope}
\definecolor{eps2pgf_color}{gray}{0}\pgfsetstrokecolor{eps2pgf_color}\pgfsetfillcolor{eps2pgf_color}
\pgftext[x=10.168cm,y=10.052cm,rotate=0]{\fontsize{27.1}{32.52}\selectfont{\textrm{\textbf{P}}}}
\end{pgfscope}
\begin{pgfscope}
\definecolor{eps2pgf_color}{gray}{0}\pgfsetstrokecolor{eps2pgf_color}\pgfsetfillcolor{eps2pgf_color}
\pgftext[x=10.544cm,y=9.833cm,rotate=0]{\fontsize{21.68}{26.02}\selectfont{\textrm{\textbf{1}}}}
\end{pgfscope}
\begin{pgfscope}
\definecolor{eps2pgf_color}{gray}{0}\pgfsetstrokecolor{eps2pgf_color}\pgfsetfillcolor{eps2pgf_color}
\pgftext[x=16.518cm,y=10.052cm,rotate=0]{\fontsize{27.1}{32.52}\selectfont{\textrm{\textbf{P}}}}
\end{pgfscope}
\begin{pgfscope}
\definecolor{eps2pgf_color}{gray}{0}\pgfsetstrokecolor{eps2pgf_color}\pgfsetfillcolor{eps2pgf_color}
\pgftext[x=16.889cm,y=9.674cm,rotate=0]{\fontsize{21.68}{26.02}\selectfont{\textrm{\textbf{2}}}}
\end{pgfscope}
\begin{pgfscope}
\definecolor{eps2pgf_color}{gray}{0}\pgfsetstrokecolor{eps2pgf_color}\pgfsetfillcolor{eps2pgf_color}
\pgftext[x=22.868cm,y=9.893cm,rotate=0]{\fontsize{27.1}{32.52}\selectfont{\textrm{\textbf{P}}}}
\end{pgfscope}
\begin{pgfscope}
\definecolor{eps2pgf_color}{gray}{0}\pgfsetstrokecolor{eps2pgf_color}\pgfsetfillcolor{eps2pgf_color}
\pgftext[x=23.235cm,y=9.669cm,rotate=0]{\fontsize{21.68}{26.02}\selectfont{\textrm{\textbf{3}}}}
\end{pgfscope}
\begin{pgfscope}
\pgfsetdash{}{0cm}
\pgfsetlinewidth{0.317mm}
\definecolor{eps2pgf_color}{gray}{0}\pgfsetstrokecolor{eps2pgf_color}\pgfsetfillcolor{eps2pgf_color}
\pgfpathmoveto{\pgfqpoint{1.778cm}{8.618cm}}
\pgfpathlineto{\pgfqpoint{5.588cm}{8.618cm}}
\pgfpathlineto{\pgfqpoint{5.588cm}{7.19cm}}
\pgfpathlineto{\pgfqpoint{1.778cm}{7.19cm}}
\pgfpathclose
\pgfusepath{stroke}
\end{pgfscope}
\begin{pgfscope}
\pgfsetdash{}{0cm}
\pgfsetlinewidth{0.317mm}
\definecolor{eps2pgf_color}{gray}{0}\pgfsetstrokecolor{eps2pgf_color}\pgfsetfillcolor{eps2pgf_color}
\pgfpathmoveto{\pgfqpoint{8.445cm}{8.777cm}}
\pgfpathlineto{\pgfqpoint{12.255cm}{8.777cm}}
\pgfpathlineto{\pgfqpoint{12.255cm}{7.348cm}}
\pgfpathlineto{\pgfqpoint{8.445cm}{7.348cm}}
\pgfpathclose
\pgfusepath{stroke}
\end{pgfscope}
\begin{pgfscope}
\pgfsetdash{}{0cm}
\pgfsetlinewidth{0.317mm}
\definecolor{eps2pgf_color}{gray}{0}\pgfsetstrokecolor{eps2pgf_color}\pgfsetfillcolor{eps2pgf_color}
\pgfpathmoveto{\pgfqpoint{14.795cm}{8.777cm}}
\pgfpathlineto{\pgfqpoint{18.605cm}{8.777cm}}
\pgfpathlineto{\pgfqpoint{18.605cm}{7.348cm}}
\pgfpathlineto{\pgfqpoint{14.795cm}{7.348cm}}
\pgfpathclose
\pgfusepath{stroke}
\end{pgfscope}
\begin{pgfscope}
\pgfsetdash{}{0cm}
\pgfsetlinewidth{0.317mm}
\definecolor{eps2pgf_color}{gray}{0}\pgfsetstrokecolor{eps2pgf_color}\pgfsetfillcolor{eps2pgf_color}
\pgfpathmoveto{\pgfqpoint{21.304cm}{8.936cm}}
\pgfpathlineto{\pgfqpoint{25.114cm}{8.936cm}}
\pgfpathlineto{\pgfqpoint{25.114cm}{7.507cm}}
\pgfpathlineto{\pgfqpoint{21.304cm}{7.507cm}}
\pgfpathclose
\pgfusepath{stroke}
\end{pgfscope}
\begin{pgfscope}
\pgfsetdash{}{0cm}
\pgfsetlinewidth{0.317mm}
\definecolor{eps2pgf_color}{gray}{0}\pgfsetstrokecolor{eps2pgf_color}\pgfsetfillcolor{eps2pgf_color}
\pgfpathmoveto{\pgfqpoint{4.477cm}{8.301cm}}
\pgfpathlineto{\pgfqpoint{5.27cm}{8.301cm}}
\pgfpathlineto{\pgfqpoint{5.27cm}{7.507cm}}
\pgfpathlineto{\pgfqpoint{4.477cm}{7.507cm}}
\pgfpathclose
\pgfusepath{stroke}
\end{pgfscope}
\begin{pgfscope}
\pgfsetdash{}{0cm}
\pgfsetlinewidth{0.317mm}
\definecolor{eps2pgf_color}{gray}{0}\pgfsetstrokecolor{eps2pgf_color}\pgfsetfillcolor{eps2pgf_color}
\pgfpathmoveto{\pgfqpoint{3.683cm}{8.301cm}}
\pgfpathlineto{\pgfqpoint{4.477cm}{8.301cm}}
\pgfpathlineto{\pgfqpoint{4.477cm}{7.507cm}}
\pgfpathlineto{\pgfqpoint{3.683cm}{7.507cm}}
\pgfpathclose
\pgfusepath{stroke}
\end{pgfscope}
\begin{pgfscope}
\pgfsetdash{}{0cm}
\pgfsetlinewidth{0.317mm}
\definecolor{eps2pgf_color}{gray}{0}\pgfsetstrokecolor{eps2pgf_color}\pgfsetfillcolor{eps2pgf_color}
\pgfpathmoveto{\pgfqpoint{2.889cm}{8.301cm}}
\pgfpathlineto{\pgfqpoint{3.683cm}{8.301cm}}
\pgfpathlineto{\pgfqpoint{3.683cm}{7.507cm}}
\pgfpathlineto{\pgfqpoint{2.889cm}{7.507cm}}
\pgfpathclose
\pgfusepath{stroke}
\end{pgfscope}
\begin{pgfscope}
\pgfsetdash{}{0cm}
\pgfsetlinewidth{0.317mm}
\definecolor{eps2pgf_color}{gray}{0}\pgfsetstrokecolor{eps2pgf_color}\pgfsetfillcolor{eps2pgf_color}
\pgfpathmoveto{\pgfqpoint{9.557cm}{8.46cm}}
\pgfpathlineto{\pgfqpoint{10.35cm}{8.46cm}}
\pgfpathlineto{\pgfqpoint{10.35cm}{7.666cm}}
\pgfpathlineto{\pgfqpoint{9.557cm}{7.666cm}}
\pgfpathclose
\pgfusepath{stroke}
\end{pgfscope}
\begin{pgfscope}
\pgfsetdash{}{0cm}
\pgfsetlinewidth{0.317mm}
\definecolor{eps2pgf_color}{gray}{0}\pgfsetstrokecolor{eps2pgf_color}\pgfsetfillcolor{eps2pgf_color}
\pgfpathmoveto{\pgfqpoint{10.35cm}{8.46cm}}
\pgfpathlineto{\pgfqpoint{11.144cm}{8.46cm}}
\pgfpathlineto{\pgfqpoint{11.144cm}{7.666cm}}
\pgfpathlineto{\pgfqpoint{10.35cm}{7.666cm}}
\pgfpathclose
\pgfusepath{stroke}
\end{pgfscope}
\begin{pgfscope}
\pgfsetdash{}{0cm}
\pgfsetlinewidth{0.317mm}
\definecolor{eps2pgf_color}{gray}{0}\pgfsetstrokecolor{eps2pgf_color}\pgfsetfillcolor{eps2pgf_color}
\pgfpathmoveto{\pgfqpoint{11.144cm}{8.46cm}}
\pgfpathlineto{\pgfqpoint{11.938cm}{8.46cm}}
\pgfpathlineto{\pgfqpoint{11.938cm}{7.666cm}}
\pgfpathlineto{\pgfqpoint{11.144cm}{7.666cm}}
\pgfpathclose
\pgfusepath{stroke}
\end{pgfscope}
\begin{pgfscope}
\pgfsetdash{}{0cm}
\pgfsetlinewidth{0.317mm}
\definecolor{eps2pgf_color}{gray}{0}\pgfsetstrokecolor{eps2pgf_color}\pgfsetfillcolor{eps2pgf_color}
\pgfpathmoveto{\pgfqpoint{15.907cm}{8.46cm}}
\pgfpathlineto{\pgfqpoint{16.7cm}{8.46cm}}
\pgfpathlineto{\pgfqpoint{16.7cm}{7.666cm}}
\pgfpathlineto{\pgfqpoint{15.907cm}{7.666cm}}
\pgfpathclose
\pgfusepath{stroke}
\end{pgfscope}
\begin{pgfscope}
\pgfsetdash{}{0cm}
\pgfsetlinewidth{0.317mm}
\definecolor{eps2pgf_color}{gray}{0}\pgfsetstrokecolor{eps2pgf_color}\pgfsetfillcolor{eps2pgf_color}
\pgfpathmoveto{\pgfqpoint{16.7cm}{8.46cm}}
\pgfpathlineto{\pgfqpoint{17.494cm}{8.46cm}}
\pgfpathlineto{\pgfqpoint{17.494cm}{7.666cm}}
\pgfpathlineto{\pgfqpoint{16.7cm}{7.666cm}}
\pgfpathclose
\pgfusepath{stroke}
\end{pgfscope}
\begin{pgfscope}
\pgfsetdash{}{0cm}
\pgfsetlinewidth{0.317mm}
\definecolor{eps2pgf_color}{gray}{0}\pgfsetstrokecolor{eps2pgf_color}\pgfsetfillcolor{eps2pgf_color}
\pgfpathmoveto{\pgfqpoint{17.494cm}{8.46cm}}
\pgfpathlineto{\pgfqpoint{18.288cm}{8.46cm}}
\pgfpathlineto{\pgfqpoint{18.288cm}{7.666cm}}
\pgfpathlineto{\pgfqpoint{17.494cm}{7.666cm}}
\pgfpathclose
\pgfusepath{stroke}
\end{pgfscope}
\begin{pgfscope}
\pgfsetdash{}{0cm}
\pgfsetlinewidth{0.317mm}
\definecolor{eps2pgf_color}{gray}{0}\pgfsetstrokecolor{eps2pgf_color}\pgfsetfillcolor{eps2pgf_color}
\pgfpathmoveto{\pgfqpoint{22.415cm}{8.618cm}}
\pgfpathlineto{\pgfqpoint{23.209cm}{8.618cm}}
\pgfpathlineto{\pgfqpoint{23.209cm}{7.825cm}}
\pgfpathlineto{\pgfqpoint{22.415cm}{7.825cm}}
\pgfpathclose
\pgfusepath{stroke}
\end{pgfscope}
\begin{pgfscope}
\pgfsetdash{}{0cm}
\pgfsetlinewidth{0.317mm}
\definecolor{eps2pgf_color}{gray}{0}\pgfsetstrokecolor{eps2pgf_color}\pgfsetfillcolor{eps2pgf_color}
\pgfpathmoveto{\pgfqpoint{23.209cm}{8.618cm}}
\pgfpathlineto{\pgfqpoint{24.003cm}{8.618cm}}
\pgfpathlineto{\pgfqpoint{24.003cm}{7.825cm}}
\pgfpathlineto{\pgfqpoint{23.209cm}{7.825cm}}
\pgfpathclose
\pgfusepath{stroke}
\end{pgfscope}
\begin{pgfscope}
\pgfsetdash{}{0cm}
\pgfsetlinewidth{0.317mm}
\definecolor{eps2pgf_color}{gray}{0}\pgfsetstrokecolor{eps2pgf_color}\pgfsetfillcolor{eps2pgf_color}
\pgfpathmoveto{\pgfqpoint{24.003cm}{8.618cm}}
\pgfpathlineto{\pgfqpoint{24.797cm}{8.618cm}}
\pgfpathlineto{\pgfqpoint{24.797cm}{7.825cm}}
\pgfpathlineto{\pgfqpoint{24.003cm}{7.825cm}}
\pgfpathclose
\pgfusepath{stroke}
\end{pgfscope}
\begin{pgfscope}
\definecolor{eps2pgf_color}{gray}{0}\pgfsetstrokecolor{eps2pgf_color}\pgfsetfillcolor{eps2pgf_color}
\pgftext[x=0.899cm,y=8.695cm,rotate=0]{\fontsize{27.1}{32.52}\selectfont{\textrm{\textbf{\textit{buf}}}}}
\end{pgfscope}
\begin{pgfscope}
\definecolor{eps2pgf_color}{gray}{0}\pgfsetstrokecolor{eps2pgf_color}\pgfsetfillcolor{eps2pgf_color}
\pgftext[x=7.567cm,y=8.854cm,rotate=0]{\fontsize{27.1}{32.52}\selectfont{\textrm{\textbf{\textit{buf}}}}}
\end{pgfscope}
\begin{pgfscope}
\definecolor{eps2pgf_color}{gray}{0}\pgfsetstrokecolor{eps2pgf_color}\pgfsetfillcolor{eps2pgf_color}
\pgftext[x=13.917cm,y=8.854cm,rotate=0]{\fontsize{27.1}{32.52}\selectfont{\textrm{\textbf{\textit{buf}}}}}
\end{pgfscope}
\begin{pgfscope}
\definecolor{eps2pgf_color}{gray}{0}\pgfsetstrokecolor{eps2pgf_color}\pgfsetfillcolor{eps2pgf_color}
\pgftext[x=20.425cm,y=9.012cm,rotate=0]{\fontsize{27.1}{32.52}\selectfont{\textrm{\textbf{\textit{buf}}}}}
\end{pgfscope}
\begin{pgfscope}
\definecolor{eps2pgf_color}{rgb}{1,0,0}\pgfsetstrokecolor{eps2pgf_color}\pgfsetfillcolor{eps2pgf_color}
\pgftext[x=0.93cm,y=9.58cm,rotate=0]{\fontsize{27.1}{32.52}\selectfont{\textrm{\textbf{\textit{send}}}}}
\end{pgfscope}
\begin{pgfscope}
\definecolor{eps2pgf_color}{rgb}{1,0,0}\pgfsetstrokecolor{eps2pgf_color}\pgfsetfillcolor{eps2pgf_color}
\pgftext[x=7.598cm,y=9.58cm,rotate=0]{\fontsize{27.1}{32.52}\selectfont{\textrm{\textbf{\textit{send}}}}}
\end{pgfscope}
\begin{pgfscope}
\definecolor{eps2pgf_color}{rgb}{1,0,0}\pgfsetstrokecolor{eps2pgf_color}\pgfsetfillcolor{eps2pgf_color}
\pgftext[x=13.948cm,y=9.58cm,rotate=0]{\fontsize{27.1}{32.52}\selectfont{\textrm{\textbf{\textit{send}}}}}
\end{pgfscope}
\begin{pgfscope}
\definecolor{eps2pgf_color}{rgb}{1,0,0}\pgfsetstrokecolor{eps2pgf_color}\pgfsetfillcolor{eps2pgf_color}
\pgftext[x=20.457cm,y=9.739cm,rotate=0]{\fontsize{27.1}{32.52}\selectfont{\textrm{\textbf{\textit{send}}}}}
\end{pgfscope}
\begin{pgfscope}
\definecolor{eps2pgf_color}{gray}{0}\pgfsetstrokecolor{eps2pgf_color}\pgfsetfillcolor{eps2pgf_color}
\pgftext[x=3.659cm,y=2.432cm,rotate=0]{\fontsize{27.1}{32.52}\selectfont{\textrm{\textbf{P}}}}
\end{pgfscope}
\begin{pgfscope}
\definecolor{eps2pgf_color}{gray}{0}\pgfsetstrokecolor{eps2pgf_color}\pgfsetfillcolor{eps2pgf_color}
\pgftext[x=4.032cm,y=2.049cm,rotate=0]{\fontsize{21.68}{26.02}\selectfont{\textrm{\textbf{0}}}}
\end{pgfscope}
\begin{pgfscope}
\definecolor{eps2pgf_color}{gray}{0}\pgfsetstrokecolor{eps2pgf_color}\pgfsetfillcolor{eps2pgf_color}
\pgftext[x=10.326cm,y=2.432cm,rotate=0]{\fontsize{27.1}{32.52}\selectfont{\textrm{\textbf{P}}}}
\end{pgfscope}
\begin{pgfscope}
\definecolor{eps2pgf_color}{gray}{0}\pgfsetstrokecolor{eps2pgf_color}\pgfsetfillcolor{eps2pgf_color}
\pgftext[x=10.702cm,y=2.213cm,rotate=0]{\fontsize{21.68}{26.02}\selectfont{\textrm{\textbf{1}}}}
\end{pgfscope}
\begin{pgfscope}
\definecolor{eps2pgf_color}{gray}{0}\pgfsetstrokecolor{eps2pgf_color}\pgfsetfillcolor{eps2pgf_color}
\pgftext[x=16.676cm,y=2.432cm,rotate=0]{\fontsize{27.1}{32.52}\selectfont{\textrm{\textbf{P}}}}
\end{pgfscope}
\begin{pgfscope}
\definecolor{eps2pgf_color}{gray}{0}\pgfsetstrokecolor{eps2pgf_color}\pgfsetfillcolor{eps2pgf_color}
\pgftext[x=17.048cm,y=2.054cm,rotate=0]{\fontsize{21.68}{26.02}\selectfont{\textrm{\textbf{2}}}}
\end{pgfscope}
\begin{pgfscope}
\definecolor{eps2pgf_color}{gray}{0}\pgfsetstrokecolor{eps2pgf_color}\pgfsetfillcolor{eps2pgf_color}
\pgftext[x=23.026cm,y=2.273cm,rotate=0]{\fontsize{27.1}{32.52}\selectfont{\textrm{\textbf{P}}}}
\end{pgfscope}
\begin{pgfscope}
\definecolor{eps2pgf_color}{gray}{0}\pgfsetstrokecolor{eps2pgf_color}\pgfsetfillcolor{eps2pgf_color}
\pgftext[x=23.394cm,y=2.049cm,rotate=0]{\fontsize{21.68}{26.02}\selectfont{\textrm{\textbf{3}}}}
\end{pgfscope}
\begin{pgfscope}
\pgfpathmoveto{\pgfqpoint{0cm}{10.407cm}}
\pgfpathlineto{\pgfqpoint{0cm}{0cm}}
\pgfpathlineto{\pgfqpoint{25.329cm}{0cm}}
\pgfpathlineto{\pgfqpoint{25.329cm}{10.407cm}}
\pgfpathclose
\pgfpathmoveto{\pgfqpoint{12.311cm}{3.007cm}}
\pgfpathlineto{\pgfqpoint{12.203cm}{3.007cm}}
\pgfpathlineto{\pgfqpoint{12.001cm}{4.283cm}}
\pgfpathlineto{\pgfqpoint{12.509cm}{4.283cm}}
\pgfpathclose
\pgfseteorule\pgfusepath{clip}\pgfsetnonzerorule
\begin{pgfscope}
\pgfsetdash{}{0cm}
\pgfsetlinewidth{0.952mm}
\definecolor{eps2pgf_color}{gray}{0}\pgfsetstrokecolor{eps2pgf_color}\pgfsetfillcolor{eps2pgf_color}
\pgfpathmoveto{\pgfqpoint{12.255cm}{6.872cm}}
\pgfpathlineto{\pgfqpoint{12.255cm}{3.062cm}}
\pgfusepath{stroke}
\end{pgfscope}
\end{pgfscope}
\begin{pgfscope}
\definecolor{eps2pgf_color}{gray}{0}\pgfsetstrokecolor{eps2pgf_color}\pgfsetfillcolor{eps2pgf_color}
\pgfpathmoveto{\pgfqpoint{12.001cm}{4.283cm}}
\pgfpathlineto{\pgfqpoint{12.255cm}{3.267cm}}
\pgfpathlineto{\pgfqpoint{12.509cm}{4.283cm}}
\pgfpathlineto{\pgfqpoint{12.001cm}{4.283cm}}
\pgfpathclose
\pgfseteorule\pgfusepath{fill}\pgfsetnonzerorule
\end{pgfscope}
\pgfsetdash{}{0cm}
\pgfsetlinewidth{0.952mm}
\definecolor{eps2pgf_color}{gray}{0}\pgfsetstrokecolor{eps2pgf_color}\pgfsetfillcolor{eps2pgf_color}
\pgfpathmoveto{\pgfqpoint{12.001cm}{4.283cm}}
\pgfpathlineto{\pgfqpoint{12.255cm}{3.267cm}}
\pgfpathlineto{\pgfqpoint{12.509cm}{4.283cm}}
\pgfpathlineto{\pgfqpoint{12.001cm}{4.283cm}}
\pgfpathclose
\pgfusepath{stroke}
\begin{pgfscope}
\pgfsetdash{}{0cm}
\pgfsetlinewidth{0.317mm}
\pgfpathmoveto{\pgfqpoint{8.763cm}{1.633cm}}
\pgfpathlineto{\pgfqpoint{12.573cm}{1.633cm}}
\pgfpathlineto{\pgfqpoint{12.573cm}{0.205cm}}
\pgfpathlineto{\pgfqpoint{8.763cm}{0.205cm}}
\pgfpathclose
\pgfusepath{stroke}
\end{pgfscope}
\begin{pgfscope}
\pgfsetdash{}{0cm}
\pgfsetlinewidth{0.317mm}
\pgfpathmoveto{\pgfqpoint{15.113cm}{1.633cm}}
\pgfpathlineto{\pgfqpoint{18.923cm}{1.633cm}}
\pgfpathlineto{\pgfqpoint{18.923cm}{0.205cm}}
\pgfpathlineto{\pgfqpoint{15.113cm}{0.205cm}}
\pgfpathclose
\pgfusepath{stroke}
\end{pgfscope}
\begin{pgfscope}
\pgfsetdash{}{0cm}
\pgfsetlinewidth{0.317mm}
\pgfpathmoveto{\pgfqpoint{21.463cm}{1.633cm}}
\pgfpathlineto{\pgfqpoint{25.273cm}{1.633cm}}
\pgfpathlineto{\pgfqpoint{25.273cm}{0.205cm}}
\pgfpathlineto{\pgfqpoint{21.463cm}{0.205cm}}
\pgfpathclose
\pgfusepath{stroke}
\end{pgfscope}
\begin{pgfscope}
\pgfsetdash{}{0cm}
\pgfsetlinewidth{0.317mm}
\pgfpathmoveto{\pgfqpoint{9.08cm}{1.316cm}}
\pgfpathlineto{\pgfqpoint{9.874cm}{1.316cm}}
\pgfpathlineto{\pgfqpoint{9.874cm}{0.522cm}}
\pgfpathlineto{\pgfqpoint{9.08cm}{0.522cm}}
\pgfpathclose
\pgfusepath{stroke}
\end{pgfscope}
\begin{pgfscope}
\pgfsetdash{}{0cm}
\pgfsetlinewidth{0.317mm}
\pgfpathmoveto{\pgfqpoint{9.874cm}{1.316cm}}
\pgfpathlineto{\pgfqpoint{10.668cm}{1.316cm}}
\pgfpathlineto{\pgfqpoint{10.668cm}{0.522cm}}
\pgfpathlineto{\pgfqpoint{9.874cm}{0.522cm}}
\pgfpathclose
\pgfusepath{stroke}
\end{pgfscope}
\begin{pgfscope}
\pgfsetdash{}{0cm}
\pgfsetlinewidth{0.317mm}
\pgfpathmoveto{\pgfqpoint{10.668cm}{1.316cm}}
\pgfpathlineto{\pgfqpoint{11.462cm}{1.316cm}}
\pgfpathlineto{\pgfqpoint{11.462cm}{0.522cm}}
\pgfpathlineto{\pgfqpoint{10.668cm}{0.522cm}}
\pgfpathclose
\pgfusepath{stroke}
\end{pgfscope}
\begin{pgfscope}
\pgfsetdash{}{0cm}
\pgfsetlinewidth{0.317mm}
\pgfpathmoveto{\pgfqpoint{11.462cm}{1.316cm}}
\pgfpathlineto{\pgfqpoint{12.255cm}{1.316cm}}
\pgfpathlineto{\pgfqpoint{12.255cm}{0.522cm}}
\pgfpathlineto{\pgfqpoint{11.462cm}{0.522cm}}
\pgfpathclose
\pgfusepath{stroke}
\end{pgfscope}
\begin{pgfscope}
\pgfsetdash{}{0cm}
\pgfsetlinewidth{0.317mm}
\pgfpathmoveto{\pgfqpoint{15.43cm}{1.316cm}}
\pgfpathlineto{\pgfqpoint{16.224cm}{1.316cm}}
\pgfpathlineto{\pgfqpoint{16.224cm}{0.522cm}}
\pgfpathlineto{\pgfqpoint{15.43cm}{0.522cm}}
\pgfpathclose
\pgfusepath{stroke}
\end{pgfscope}
\begin{pgfscope}
\pgfsetdash{}{0cm}
\pgfsetlinewidth{0.317mm}
\pgfpathmoveto{\pgfqpoint{16.224cm}{1.316cm}}
\pgfpathlineto{\pgfqpoint{17.018cm}{1.316cm}}
\pgfpathlineto{\pgfqpoint{17.018cm}{0.522cm}}
\pgfpathlineto{\pgfqpoint{16.224cm}{0.522cm}}
\pgfpathclose
\pgfusepath{stroke}
\end{pgfscope}
\begin{pgfscope}
\pgfsetdash{}{0cm}
\pgfsetlinewidth{0.317mm}
\pgfpathmoveto{\pgfqpoint{17.018cm}{1.316cm}}
\pgfpathlineto{\pgfqpoint{17.812cm}{1.316cm}}
\pgfpathlineto{\pgfqpoint{17.812cm}{0.522cm}}
\pgfpathlineto{\pgfqpoint{17.018cm}{0.522cm}}
\pgfpathclose
\pgfusepath{stroke}
\end{pgfscope}
\begin{pgfscope}
\pgfsetdash{}{0cm}
\pgfsetlinewidth{0.317mm}
\pgfpathmoveto{\pgfqpoint{17.812cm}{1.316cm}}
\pgfpathlineto{\pgfqpoint{18.605cm}{1.316cm}}
\pgfpathlineto{\pgfqpoint{18.605cm}{0.522cm}}
\pgfpathlineto{\pgfqpoint{17.812cm}{0.522cm}}
\pgfpathclose
\pgfusepath{stroke}
\end{pgfscope}
\begin{pgfscope}
\pgfsetdash{}{0cm}
\pgfsetlinewidth{0.317mm}
\pgfpathmoveto{\pgfqpoint{21.78cm}{1.316cm}}
\pgfpathlineto{\pgfqpoint{22.574cm}{1.316cm}}
\pgfpathlineto{\pgfqpoint{22.574cm}{0.522cm}}
\pgfpathlineto{\pgfqpoint{21.78cm}{0.522cm}}
\pgfpathclose
\pgfusepath{stroke}
\end{pgfscope}
\begin{pgfscope}
\pgfsetdash{}{0cm}
\pgfsetlinewidth{0.317mm}
\pgfpathmoveto{\pgfqpoint{22.574cm}{1.316cm}}
\pgfpathlineto{\pgfqpoint{23.368cm}{1.316cm}}
\pgfpathlineto{\pgfqpoint{23.368cm}{0.522cm}}
\pgfpathlineto{\pgfqpoint{22.574cm}{0.522cm}}
\pgfpathclose
\pgfusepath{stroke}
\end{pgfscope}
\begin{pgfscope}
\pgfsetdash{}{0cm}
\pgfsetlinewidth{0.317mm}
\pgfpathmoveto{\pgfqpoint{23.368cm}{1.316cm}}
\pgfpathlineto{\pgfqpoint{24.162cm}{1.316cm}}
\pgfpathlineto{\pgfqpoint{24.162cm}{0.522cm}}
\pgfpathlineto{\pgfqpoint{23.368cm}{0.522cm}}
\pgfpathclose
\pgfusepath{stroke}
\end{pgfscope}
\begin{pgfscope}
\pgfsetdash{}{0cm}
\pgfsetlinewidth{0.317mm}
\pgfpathmoveto{\pgfqpoint{24.162cm}{1.316cm}}
\pgfpathlineto{\pgfqpoint{24.955cm}{1.316cm}}
\pgfpathlineto{\pgfqpoint{24.955cm}{0.522cm}}
\pgfpathlineto{\pgfqpoint{24.162cm}{0.522cm}}
\pgfpathclose
\pgfusepath{stroke}
\end{pgfscope}
\begin{pgfscope}
\pgfsetdash{}{0cm}
\pgfsetlinewidth{0.317mm}
\pgfpathmoveto{\pgfqpoint{2.095cm}{1.475cm}}
\pgfpathlineto{\pgfqpoint{5.905cm}{1.475cm}}
\pgfpathlineto{\pgfqpoint{5.905cm}{0.046cm}}
\pgfpathlineto{\pgfqpoint{2.095cm}{0.046cm}}
\pgfpathclose
\pgfusepath{stroke}
\end{pgfscope}
\begin{pgfscope}
\pgfsetdash{}{0cm}
\pgfsetlinewidth{0.317mm}
\pgfpathmoveto{\pgfqpoint{4.794cm}{1.157cm}}
\pgfpathlineto{\pgfqpoint{5.588cm}{1.157cm}}
\pgfpathlineto{\pgfqpoint{5.588cm}{0.363cm}}
\pgfpathlineto{\pgfqpoint{4.794cm}{0.363cm}}
\pgfpathclose
\pgfusepath{stroke}
\end{pgfscope}
\begin{pgfscope}
\pgfsetdash{}{0cm}
\pgfsetlinewidth{0.317mm}
\pgfpathmoveto{\pgfqpoint{4cm}{1.157cm}}
\pgfpathlineto{\pgfqpoint{4.794cm}{1.157cm}}
\pgfpathlineto{\pgfqpoint{4.794cm}{0.363cm}}
\pgfpathlineto{\pgfqpoint{4cm}{0.363cm}}
\pgfpathclose
\pgfusepath{stroke}
\end{pgfscope}
\begin{pgfscope}
\pgfsetdash{}{0cm}
\pgfsetlinewidth{0.317mm}
\pgfpathmoveto{\pgfqpoint{3.207cm}{1.157cm}}
\pgfpathlineto{\pgfqpoint{4cm}{1.157cm}}
\pgfpathlineto{\pgfqpoint{4cm}{0.363cm}}
\pgfpathlineto{\pgfqpoint{3.207cm}{0.363cm}}
\pgfpathclose
\pgfusepath{stroke}
\end{pgfscope}
\begin{pgfscope}
\pgfpathmoveto{\pgfqpoint{0cm}{10.407cm}}
\pgfpathlineto{\pgfqpoint{0cm}{0cm}}
\pgfpathlineto{\pgfqpoint{25.329cm}{0cm}}
\pgfpathlineto{\pgfqpoint{25.329cm}{10.407cm}}
\pgfpathclose
\pgfpathmoveto{\pgfqpoint{3.177cm}{1.269cm}}
\pgfpathlineto{\pgfqpoint{3.16cm}{1.345cm}}
\pgfpathlineto{\pgfqpoint{4.043cm}{1.699cm}}
\pgfpathlineto{\pgfqpoint{4.125cm}{1.329cm}}
\pgfpathclose
\pgfseteorule\pgfusepath{clip}\pgfsetnonzerorule
\begin{pgfscope}
\pgfsetdash{{0.254cm}}{0cm}
\pgfsetlinewidth{0.635mm}
\pgfpathmoveto{\pgfqpoint{17.97cm}{4.65cm}}
\pgfpathlineto{\pgfqpoint{3.207cm}{1.316cm}}
\pgfusepath{stroke}
\end{pgfscope}
\end{pgfscope}
\begin{pgfscope}
\pgfpathmoveto{\pgfqpoint{4.043cm}{1.699cm}}
\pgfpathlineto{\pgfqpoint{3.34cm}{1.345cm}}
\pgfpathlineto{\pgfqpoint{4.125cm}{1.329cm}}
\pgfpathlineto{\pgfqpoint{4.043cm}{1.699cm}}
\pgfpathclose
\pgfseteorule\pgfusepath{fill}\pgfsetnonzerorule
\end{pgfscope}
\pgfsetdash{}{0cm}
\pgfsetlinewidth{0.635mm}
\pgfpathmoveto{\pgfqpoint{4.043cm}{1.699cm}}
\pgfpathlineto{\pgfqpoint{3.34cm}{1.345cm}}
\pgfpathlineto{\pgfqpoint{4.125cm}{1.329cm}}
\pgfpathlineto{\pgfqpoint{4.043cm}{1.699cm}}
\pgfpathclose
\pgfusepath{stroke}
\begin{pgfscope}
\definecolor{eps2pgf_color}{rgb}{1,0,0}\pgfsetstrokecolor{eps2pgf_color}\pgfsetfillcolor{eps2pgf_color}
\pgftext[x=13.433cm,y=5.053cm,rotate=0]{\fontsize{27.1}{32.52}\selectfont{\textrm{\textbf{MPI{\_}Reduce(sendbuf, ..., recvbuf, .., MPI{\_}OP, 0, ... );}}}}
\end{pgfscope}
\begin{pgfscope}
\pgftext[x=7.884cm,y=1.71cm,rotate=0]{\fontsize{27.1}{32.52}\selectfont{\textrm{\textbf{\textit{buf}}}}}
\end{pgfscope}
\begin{pgfscope}
\pgftext[x=14.234cm,y=1.71cm,rotate=0]{\fontsize{27.1}{32.52}\selectfont{\textrm{\textbf{\textit{buf}}}}}
\end{pgfscope}
\begin{pgfscope}
\pgftext[x=20.584cm,y=1.71cm,rotate=0]{\fontsize{27.1}{32.52}\selectfont{\textrm{\textbf{\textit{buf}}}}}
\end{pgfscope}
\begin{pgfscope}
\pgftext[x=1.217cm,y=1.551cm,rotate=0]{\fontsize{27.1}{32.52}\selectfont{\textrm{\textbf{\textit{buf}}}}}
\end{pgfscope}
\begin{pgfscope}
\definecolor{eps2pgf_color}{rgb}{1,0,0}\pgfsetstrokecolor{eps2pgf_color}\pgfsetfillcolor{eps2pgf_color}
\pgftext[x=1.138cm,y=2.323cm,rotate=0]{\fontsize{27.1}{32.52}\selectfont{\textrm{\textbf{\textit{recv}}}}}
\end{pgfscope}
\begin{pgfscope}
\definecolor{eps2pgf_color}{rgb}{1,0,0}\pgfsetstrokecolor{eps2pgf_color}\pgfsetfillcolor{eps2pgf_color}
\pgftext[x=7.488cm,y=2.323cm,rotate=0]{\fontsize{27.1}{32.52}\selectfont{\textrm{\textbf{\textit{recv}}}}}
\end{pgfscope}
\begin{pgfscope}
\definecolor{eps2pgf_color}{rgb}{1,0,0}\pgfsetstrokecolor{eps2pgf_color}\pgfsetfillcolor{eps2pgf_color}
\pgftext[x=13.838cm,y=2.323cm,rotate=0]{\fontsize{27.1}{32.52}\selectfont{\textrm{\textbf{\textit{recv}}}}}
\end{pgfscope}
\begin{pgfscope}
\definecolor{eps2pgf_color}{rgb}{1,0,0}\pgfsetstrokecolor{eps2pgf_color}\pgfsetfillcolor{eps2pgf_color}
\pgftext[x=20.347cm,y=2.323cm,rotate=0]{\fontsize{27.1}{32.52}\selectfont{\textrm{\textbf{\textit{recv}}}}}
\end{pgfscope}
\end{pgfscope}
\end{pgfscope}
\end{pgfpicture}

    }
    \caption{Visualización de MPI\_Reduce \cite{cheung_mpi}}
    \label{fig:mpi_reduce}
  \end{figure}

\end{itemize}

%%%% TODO INSERTAR FIGURAS CON TIKZ

