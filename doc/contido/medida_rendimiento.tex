\chapter{Medida del rendimiento}
\label{chap:medida_rendimiento}

\lettrine{L}{a} medida del rendimiento y el resultado de la misma es una parte fundamental en la evaluación y relevancia de un supercomputador, hasta el punto en el que se compite por ver cuales desarrollan un mejor resultado\footnote{\url{https://www.top500.org}}.

En este caso, si bien el rendimiento no es una prioridad, es conveniente realizar estas pruebas, especialmente para poder observar el impacto que tiene la red de comunicaciones entre los núcleos de una sola CPU, o entre múltiples CPUs y memorias. En resumen, medir la escalabilidad.

\section{NAS Parallel Benchmarks}
Los \acrlong{npb} (\acrshort{npb})\footnote{\url{https://www.nas.nasa.gov/software/npb.html}} son una \textit{suite} de tests diseñados por la División de Supercomputación de la NASA para la medida del rendimiento de supercomputadores paralelos.

Estos benchmarks se dividen principalmente ocho benchmarks originales, teniendo otras adaptaciones para diferentes paradigmas de computación paralela, como puede ser de paralelización híbrida, computación desestructurada, o mallas de computación.

A su vez cada benchmark tiene múltiples clases, siendo estas \texttt{S, W, A, B, C, D, E}, y representando cada una un nivel de requerimientos superior, respectivamente.

En este caso se empleará la clase C para todos los benchmarks debido a limitaciones de memoria en los nodos, menos en el benchmark DT, que únicamente permite hasta clase B.

\subsection{Prerrequisitos}
Para poder realizar las medidas de rendimiento, primero se necesitan tanto las herramientas de compilación como las de ejecución. Estas dependencias se deben satisfacer ejecutando:

\begin{lstlisting}[language=bash]
# Se añade la opción --needed debido a que hay paquetes del metapaquete base-devel que ya están instalados, y no es necesario reinstalar.
# Tras ejecutar el comando se aceptan las opciones por defecto
pacman -S base-devel openmpi gcc-fortran wget --needed
\end{lstlisting}

\subsection{Instalación de los \acrshort{npb}}
A continuación se deben descargar, configurar, y compilar los \acrlong{npb} únicamente en el nodo maestro. Esto se realizará como \texttt{mpiuser} y en la carpeta \texttt{/mpishared} de la siguiente forma:

\begin{lstlisting}[language=bash]
# Descarga de la última versión de NPB (3.4.2)
wget https://www.nas.nasa.gov/assets/npb/NPB3.4.2.tar.gz

# Descomprimimos el archivo descargado
tar xzfv NPB3.4.2.tar.gz

# Y entramos en la versión de los NPB con MPI
cd /mpishared/NPB3.4.2/NPB3.4-MPI

# Copiamos los archivos de configuración para la compilación
#  1- Para el archivo make.def añadimos un par de argumentos a los flags de compilación:
#   1.1- -fallow-argument-mismatch Esto permite que el código compile en las últimas versiones de gfortran
#   1.2- -march=native Optimiza para la arquitectura específica de rpi4
sed 's/^FFLAGS\s*=\s*-O3/FFLAGS  = -O3 -fallow-argument-mismatch -march=native/g' config/make.def.template | sed -e 's/^CFLAGS\s*=\s*-O3/CFLAGS  = -O3 -march=native/g' > config/make.def

#  2- Para el archivo suite.def hacemos que todos los tests sean clase C menos DT, que debe ser B
sed 's/S$/C/g' config/suite.def.template | sed -e 's/dt\s*C$/dt\tB/g' > config/suite.def

# Finalmente compilamos todos los tests con make suite (con todas las CPU en paralelo)
make suite -j4
\end{lstlisting}

\subsection{Preparativos para la ejecución de los benchmarks}
Para la ejecución de los benchmarks de forma paralela se necesita algún método para coordinar la ejecución del programa en todos los nodos.

\begin{itemize}
    \item Para un solo nodo con múltiples núcleos es sencillo, ya que comparten memoria, usuarios y almacenamiento
    \item Para múltiples nodos sin embargo es algo más complejo. Necesitamos
    \begin{itemize}
        \item Almacenamiento compartido en forma de NFS para alojar los ejecutables
        \item Autenticación y gestión común para el lanzamiento de los programas paralelos en forma de SSH con par de claves
    \end{itemize}
\end{itemize}

A MPI se le indica qué nodos deben ejecutar las tareas mediante un archivo de hosts, o \textit{hostsfile}\cite{mpi_hostfile_option}. Este fichero contiene las direcciones de los diferentes computadores que componen el cluster, así como el número de tareas que pueden ejecutar (habitualmente coincidente con su número de núcleos)

Así, crearemos en \texttt{/mpishared} el fichero \texttt{hostfile} con los hostname de todos los nodos:

\begin{lstlisting}[language=bash]
rpi0 slots=4
rpi1 slots=4
rpi2 slots=4
rpi3 slots=4
rpi4 slots=4
rpi5 slots=4
rpi6 slots=4
rpi7 slots=4
\end{lstlisting}

\subsection{Ejecución de los benchmarks}
Ejecutar los benchmarks es moderadamente sencillo: \texttt{mpirun} funciona de la siguiente forma <insertar explicación detallada> (posiblemente reubicar)

La ejecución de cada uno de los benchmarks se realizará mediante el siguiente comando \textbf{y como usuario mpiuser}, en el que deberemos especificar el número de procesos que queremos crear.

\begin{lstlisting}[language=bash]
mpirun -np <NUM_PROCS> --hostfile /mpishared/hostfile --mca opal_warn_on_missing_libcuda 0 /mpishared/NPB3.4.2/NPB3.4-MPI/bin/<KERNEL>.<CLASS>.x
\end{lstlisting}

\section{Comparación de Resultados}
TODO