\chapter{Contenido didáctico}
\label{chap:contenido_didactico}

\lettrine{U}{na} importante parte de este proyecto es la forma en la que sirve para explicar el funcionamiento del mismo, así como realizar demostraciones prácticas de dicho funcionamiento.

Para este cometido se desarrolla una aplicación web desde la cual se podrán ejecutar ....

\section{Requisitos}
Los requisitos son relacionados con docencia, utilidad, tendrán vídeos, será sencillo de usar etc......

\subsection{Requisitos funcionales y no funcionales}
Requisitos blabla

\subsection{Prerrequisitos}
- Instalar npm

\begin{lstlisting}[language=bash]
pacman -S npm
\end{lstlisting}

Una vez instalado npm, se ubica la app en (ruta) y se instalan las dependencias con
\begin{lstlisting}[language=bash]
npm install --only=production
\end{lstlisting}

Se ejecuta una vez de prueba con
\begin{lstlisting}[language=bash]
node ./index.js
\end{lstlisting}

Crear unit systemd

Activar unit

\subsection{Instalación de Netdata}
Netdata es el software de monitorización que nos permitirá obtener información del cluster en tiempo real, tanto en el \textit{Dashboard} web que se pone a disposición del usuario en el puerto 19999, como a través de la librería JavaScript que un programador puede importar en su propia página web.

Para emplear los datos que nos proporciona este software, primero debe instalarse y activarse en cada uno de los nodos del cluster, acción que se realiza con los comandos como usuario root.

\begin{lstlisting}[language=bash]
pacman -S netdata   # Se instala netdata
# Se activa netdata a través de todas las interfaces
sed -i 's/bind to = localhost/bind to = 0.0.0.0/g' /etc/netdata/netdata.conf
# Se activa e inicia netdata
systemctl enable --now netdata

\end{lstlisting}


\section{Diseño}
Se sigue un proceso de diseño mediante el uso de prototipos (\textit{mockups})... (TODO insertar escaneos de mockups en diseño)

\section{Funcionamiento}
