\chapter{Introducción}
\label{chap:introducion}

\lettrine{E}{n} este capítulo se expondrá qué motivó este trabajo y sus objetivos, así como qué estructura seguirá la memoria del mismo.

%Ejemplo: capítulo da memoria, onde xeralmente se exporán as
%liñas mestras do traballo, os obxectivos, etc. Incluimos un par de
%exemplos de citas~\cite{ErlangBook,ErlangWebBook} e de referencias
%internas (sección \ref{sec:mostra}, páxina \pageref{sec:mostra}).

\section{Motivación}
\label{sec:motivacion}

La supercomputación y la inteligencia artificial están de moda. Quizás más la segunda que la primera, pero lo que es cierto es que sin base sobre la que ejecutarse, la segunda no tiene mucho que hacer.
Los supercomputadores son equipos informáticos compuestos por miles de procesadores, así como cantidades ingentes de memoria para ofrecer una elevada capacidad y velocidad de cálculo y procesamiento de datos.
Sin embargo, estas máquinas y sus mecanismos y procesos tan potentes suelen quedar no solo muchas veces fuera de la comprensión del público general, sino incluso de las personas que tienen la informática como \textit{hobby} o profesión.

Para no quedarse atrás, la Unión Europea está realizando una muy importante inversión en las dos disciplinas anterioremente mencionadas, pero especialmente en \acrshort{hpc}, donde la inversión ya supera a la de la \acrshort{ia}, como se puede ver en una hoja publicada por la Unión Europea en el Anexo \ref{chap:factsheet_europe}.
Por esta razón, este \acrshort{tfg} tiene como objetivo atraer la atención de futuros ingenieros hacia este sector clave, que tanta inversión recibe y se espera que siga recibiendo. Para ello se construirá un pequeño clúster con Raspberry Pis con el nombre de Clúpiter, con el que poder realizar explicaciones acerca de cómo funciona este tan apasionante y demandado ámbito de la informática.

El principal componente de Clúpiter, la Raspberry Pi, es un computador de formato muy reducido y bajo consumo y coste, usado muy habitualmente en proyectos amateur y entornos educativos, lo que posibilita su empleo en proyectos de divulgación como el que aquí se realiza.

\section{Objetivos}
\label{sec:objetivos}

El principal objetivo de este trabajo es acercar la supercomputación y el procesamiento paralelo a públicos no especializados de una forma didáctica y amena mediante la construcción de un clúster que pueda emular el funcionamiento de un supercomputador. Para conseguir este objetivo primero necesitaremos una base tangible con la que trabajar y sobre la que podamos realizar las explicaciones. Esto es, el hardware del clúster, que se construirá utilizando las previamente mencionadas Raspberry Pis. El sistema construido pretende ser una réplica a pequeña escala de los supercomputadores actuales.

Sobre este hardware debe correr un sistema operativo en el que ejecutar los benchmarks y demostraciones, tanto a efectos académicos como divulgativos, respectivamente. Este sistema operativo será Arch Linux, un Linux que se define como ``una distribución ligera y flexible que intenta mantener la simpleza'', y que sigue los principios de Simplicidad, Modernidad, Pragmatismo, Centralidad del Usuario y Versatilidad\footnote{\url{https://wiki.archlinux.org/title/Arch\_Linux}}.

En el transcurso del proyecto se conectarán y configurarán las Raspberry Pi para que puedan trabajar de forma colaborativa como si se tratase de un único equipo, permitiendo así realizar divulgación acerca del \acrshort{hpc} a alumnos de secundaria y bachillerato que tengan interés en la carrera, así como también al público general.

Asimismo, la base hardware y software también debe servir para futuros TFGs que mejoren o amplíen ciertas características de Clúpiter, como su mantenibilidad, seguridad, características de alta disponibilidad, etc, siendo estas de nuevo habilidades muy demandadas por la Unión Europea y que están sujetas a fuertes inversiones actuales y futuras.



\section{Estructura}
\label{sec:estructura}

TODO ESTRUCTURA
