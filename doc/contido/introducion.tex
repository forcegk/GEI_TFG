\chapter{Introducción}
\label{chap:introducion}

\lettrine{E}{n} este capítulo se expondrán qué motivó este trabajo y sus objetivos, así como qué estructura seguirá la memoria del mismo.




%Ejemplo: capítulo da memoria, onde xeralmente se exporán as
%liñas mestras do traballo, os obxectivos, etc. Incluimos un par de
%exemplos de citas~\cite{ErlangBook,ErlangWebBook} e de referencias
%internas (sección \ref{sec:mostra}, páxina \pageref{sec:mostra}).

\section{Motivación}
\label{sec:motivacion}

Los supercomputadores son equipos informáticos compuestos por miles de procesadores, así como cantidades ingentes de memoria para ofrecer una elevada capacidad y velocidad de cálculo y procesamiento de datos.

Sin embargo, estas máquinas y sus mecanismos y procesos tan potentes suelen quedar no solo muchas veces fuera de la comprensión del público general, sino incluso de las personas que tienen la informática como \textit{hobby} o profesión.

Raspberry Pi es un computador de formato muy reducido y bajo consumo y coste, usado muy habitualmente en proyectos amateur y entornos educativos, y que nos posibilita su empleo en proyectos de divulgación. 

\section{Objetivos}
\label{sec:objetivos}

El principal objetivo de este trabajo es acercar la supercomputación y el procesamiento paralelo a públicos no especializados de una forma didáctica y amena mediante la construcción de un cluster paralelo que pueda emular el funcionamiento de un supercomputador.

Para conseguir este objetivo primero necesitaremos una base tangible con la que trabajar y sobre la que podamos realizar las explicaciones. Esto es, el hardware del cluster, que se construirá utilizando las previamente mencionadas Raspberry Pis. El sistema construido pretende ser una réplica a pequeña escala de los supercomputadores actuales.

Sobre ese mismo hardware debe correr un sistema operativo en el que ejecutar los benchmarks y demostraciones, tanto a efectos académicos como divulgativos, respectivamente.

En el transcurso del proyecto se conectarán y configurarán las Raspberry Pi para que puedan trabajar de forma colaborativa como si se tratase de un único equipo, permitiendo así realizar divulgación acerca del \acrshort{hpc} a alumnos de secundaria y al público general.

Asimismo, la base hardware y software también debe servir para futuros TFGs que mejoren o amplíen ciertas características del cluster, como su mantenibilidad, seguridad, características de alta disponibilidad, etc.



\section{Estructura}
\label{sec:estructura}

estruc
