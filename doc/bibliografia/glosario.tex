%%%%%%%%%%%%%%%%%%%%%%%%%%%%%%%%%%%%%%%%%%%%%%%%%%%%%%%%%%%%%%%%%%%%%%%%%%%%%%%%
% Obxectivo: Lista de termos empregados no documento,                          %
%            xunto cos seus respectivos significados.                          %
%%%%%%%%%%%%%%%%%%%%%%%%%%%%%%%%%%%%%%%%%%%%%%%%%%%%%%%%%%%%%%%%%%%%%%%%%%%%%%%%

\newglossaryentry{rootfs}{
  name=rootfs,
  description={Sistema de ficheros raíz, que contiene al resto de ficheros y directorios y por tanto permite que se arranque un sistema operativo desde los contenidos del mismo}
}

\newglossaryentry{backend}{
  name=backend,
  description={En desarrollo web es la parte que se encarga del correcto funcionamiento de la lógica de una página web. Esta lógica por lo general no es visible para el usuario (por ejemplo la comunicación con el servidor, o el acceso a datos), por lo que se dice que está en la ``parte-de-atrás'' o back-end}
}

\newglossaryentry{overclock}{
  name=overclock,
  description={El overclocking es una práctica, habitualmente realizada por usuarios entusiastas, que consiste en elevar la frecuencia de reloj de un componente por encima de las especificaciones que dictamina el fabricante}
}